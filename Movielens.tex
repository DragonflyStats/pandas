
%http://www.gregreda.com/2013/10/26/using-pandas-on-the-movielens-dataset/



\documentclass[11pt]{article} % use larger type; default would be 10pt

\usepackage[utf8]{inputenc} 
\usepackage{geometry} % to change the page dimensions
\geometry{a4paper} 
\usepackage{graphicx} 
\usepackage{booktabs} % for much better looking tables
\usepackage{array} % for better arrays (eg matrices) in maths
\usepackage{paralist} % very flexible & customisable lists (eg. enumerate/itemize, etc.)
\usepackage{verbatim} % adds environment for commenting out blocks of text & for better verbatim
\usepackage{subfig} 
\usepackage{framed}
\usepackage{subfiles}
\usepackage{fancyhdr} % This should be set AFTER setting up the page geometry
\pagestyle{fancy} % options: empty , plain , fancy
\renewcommand{\headrulewidth}{0pt} % customise the layout...
\lhead{}\chead{Data Analysis with Python}\rhead{}
\lfoot{}\cfoot{\thepage}\rfoot{}
%--------------------------------------------------------------------------------------------%
\usepackage{sectsty}
\allsectionsfont{\sffamily\mdseries\upshape} 
\usepackage[nottoc,notlof,notlot]{tocbibind} % Put the bibliography in the ToC
\usepackage[titles,subfigure]{tocloft} % Alter the style of the Table of Contents
\renewcommand{\cftsecfont}{\rmfamily\mdseries\upshape}
\renewcommand{\cftsecpagefont}{\rmfamily\mdseries\upshape} % No bold!
%--------------------------------------------------------------------------------------------%

\title{Brief Article}
\author{The Author}
%--------------------------------------------------------------------------------------------%

\begin{document}

%------------------------------------------------------------------ %

Using pandas on the MovieLens dataset

To show pandas in a more "applied" sense, let's use it to answer some questions about the MovieLens dataset. Recall that we've already read our data into DataFrames and merged it.

In [61]:
# pass in column names for each CSV
u_cols = ['user_id', 'age', 'sex', 'occupation', 'zip_code']
users = pd.read_csv('ml-100k/u.user', sep='|', names=u_cols)

r_cols = ['user_id', 'movie_id', 'rating', 'unix_timestamp']
ratings = pd.read_csv('ml-100k/u.data', sep='\t', names=r_cols)

# the movies file contains columns indicating the movie's genres
# let's only load the first five columns of the file with usecols
m_cols = ['movie_id', 'title', 'release_date', 'video_release_date', 'imdb_url']
movies = pd.read_csv('ml-100k/u.item', sep='|', names=m_cols, usecols=range(5))

# create one merged DataFrame
movie_ratings = pd.merge(movies, ratings)
lens = pd.merge(movie_ratings, users)
What are the 25 most rated movies?

\begin{verbatim}
In [62]:
most_rated = lens.groupby('title').size().order(ascending=False)[:25]
\end{verbatim}
print most_rated
title
Star Wars (1977)                             583
Contact (1997)                               509
Fargo (1996)                                 508
Return of the Jedi (1983)                    507
Liar Liar (1997)                             485
English Patient, The (1996)                  481
Scream (1996)                                478
Toy Story (1995)                             452
Air Force One (1997)                         431
Independence Day (ID4) (1996)                429
Raiders of the Lost Ark (1981)               420
Godfather, The (1972)                        413
Pulp Fiction (1994)                          394
Twelve Monkeys (1995)                        392
Silence of the Lambs, The (1991)             390
Jerry Maguire (1996)                         384
Chasing Amy (1997)                           379
Rock, The (1996)                             378
Empire Strikes Back, The (1980)              367
Star Trek: First Contact (1996)              365
Titanic (1997)                               350
Back to the Future (1985)                    350
Mission: Impossible (1996)                   344
Fugitive, The (1993)                         336
Indiana Jones and the Last Crusade (1989)    331
dtype: int64

There's a lot going on in the code above, but it's very idomatic. We're splitting the DataFrame into groups by movie title and applying the size method to get the count of records in each group. Then we order our results in descending order and limit the output to the top 25 using Python's slicing syntax.

In SQL, this would be equivalent to:

\begin{framed}
\begin{verbatim}
SELECT title, count(1)
FROM lens
GROUP BY title
ORDER BY 2 DESC
LIMIT 25;
\end{verbatim}
\end{framed}
Alternatively, pandas has a nifty value_counts method - yes, this is simpler - the goal above was to show a basic groupby example.

\begin{verbatim}
lens.title.value_counts()[:25]
Out[63]:
Star Wars (1977)                             583
Contact (1997)                               509
Fargo (1996)                                 508
Return of the Jedi (1983)                    507
Liar Liar (1997)                             485
English Patient, The (1996)                  481
Scream (1996)                                478
Toy Story (1995)                             452
Air Force One (1997)                         431
Independence Day (ID4) (1996)                429
Raiders of the Lost Ark (1981)               420
Godfather, The (1972)                        413
Pulp Fiction (1994)                          394
Twelve Monkeys (1995)                        392
Silence of the Lambs, The (1991)             390
Jerry Maguire (1996)                         384
Chasing Amy (1997)                           379
Rock, The (1996)                             378
Empire Strikes Back, The (1980)              367
Star Trek: First Contact (1996)              365
Titanic (1997)                               350
Back to the Future (1985)                    350
Mission: Impossible (1996)                   344
Fugitive, The (1993)                         336
Indiana Jones and the Last Crusade (1989)    331
dtype: int64
\end{verbatim}
Which movies are most highly rated?


\begin{framed}
\begin{verbatim}
movie_stats = lens.groupby('title').agg({'rating': [np.size, np.mean]})
print movie_stats.head()
\end{verbatim}
\end{framed}
                           rating          
                             size      mean
title                                      
'Til There Was You (1997)       9  2.333333
1-900 (1994)                    5  2.600000
101 Dalmatians (1996)         109  2.908257
12 Angry Men (1957)           125  4.344000
187 (1997)                     41  3.024390

We can use the agg method to pass a dictionary specifying the columns to aggregate (as keys) and a list of functions we'd like to apply.

Let's sort the resulting DataFrame so that we can see which movies have the highest average score.

\begin{framed}
\begin{verbatim}
# sort by rating average
print movie_stats.sort([('rating', 'mean')], ascending=False).head()
\end{verbatim}
\end{framed}


                                            rating      
                                              size  mean
title                                                   
Marlene Dietrich: Shadow and Light (1996)        1     5
Prefontaine (1997)                               3     5
Santa with Muscles (1996)                        2     5
Star Kid (1997)                                  3     5
Someone Else's America (1995)                    1     5

Because movie_stats is a DataFrame, we use the sort method - only Series objects use order. Additionally, because our columns are now a MultiIndex, we need to pass in a tuple specifying how to sort.

%------------------------------------------------------------------------- %
\end{document}
