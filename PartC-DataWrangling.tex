
\documentclass[11pt]{article} % use larger type; default would be 10pt

\usepackage[utf8]{inputenc} 
\usepackage{geometry} % to change the page dimensions
\geometry{a4paper} 
\usepackage{graphicx} 
\usepackage{booktabs} % for much better looking tables
\usepackage{array} % for better arrays (eg matrices) in maths
\usepackage{paralist} % very flexible & customisable lists (eg. enumerate/itemize, etc.)
\usepackage{verbatim} % adds environment for commenting out blocks of text & for better verbatim
\usepackage{subfig} 
\usepackage{framed}
\usepackage{subfiles}
\usepackage{fancyhdr} % This should be set AFTER setting up the page geometry
\pagestyle{fancy} % options: empty , plain , fancy
\renewcommand{\headrulewidth}{0pt} % customise the layout...
\lhead{}\chead{Data Analysis with Python}\rhead{}
\lfoot{}\cfoot{\thepage}\rfoot{}
%--------------------------------------------------------------------------------------------%
\usepackage{sectsty}
\allsectionsfont{\sffamily\mdseries\upshape} 
\usepackage[nottoc,notlof,notlot]{tocbibind} % Put the bibliography in the ToC
\usepackage[titles,subfigure]{tocloft} % Alter the style of the Table of Contents
\renewcommand{\cftsecfont}{\rmfamily\mdseries\upshape}
\renewcommand{\cftsecpagefont}{\rmfamily\mdseries\upshape} % No bold!
%--------------------------------------------------------------------------------------------%

\title{Brief Article}
\author{The Author}
%--------------------------------------------------------------------------------------------%

\begin{document}

%----------------------------------------------------------------------------%
\section{Data Wrangling: Clean, Transform, Merge, Reshape}
%-------------------------------%
\subsection{Combining and Merging Data Sets}
Database-style DataFrame Merges\\
Merging on Index\\
Concatenating Along an Axis\\
Combining Data with Overlap\\
%-------------------------------%
\subsection{Reshaping and Pivoting}
Reshaping with Hierarchical Indexing\\
Pivoting “long” to “wide” Format\\
%----------------------------------------------------------------------------%
%----------------------------------------------------------------------------%
\subsection{Data Transformation}
Removing Duplicates\\
Transforming Data Using a Function or Mapping\\
Replacing Values\\
Renaming Axis Indexes\\
Discretization and Binning\\
Detecting and Filtering Outliers\\
Permutation and Random Sampling\\
Computing Indicator/Dummy Variables\\
%----------------------------------------------------------------------------%
%----------------------------------------------------------------------------%
\subsection{String Manipulation}
String Object Methods\\
Regular expressions\\
Vectorized string functions in pandas\\

%-------------------------------%
\subsection{Example: USDA Food Database}


%----------------------------------------------------------------------------%
%----------------------------------------------------------------------------%
\newpage
\subsection*{Merge, join, and concatenate}

%- http://pandas.pydata.org/pandas-docs/stable/merging.html
pandas provides various facilities for easily combining together Series, DataFrame, and Panel objects with various kinds of set logic for the indexes and relational algebra
functionality in the case of join / merge-type operations.


%- MERGING DATA EXAMPLE
%- http://stackoverflow.com/questions/14341805/pandas-merge-pd-merge-how-to-set-the-index-and-join


Concatenating objects
The concat function (in the main pandas namespace) does all of the heavy lifting of performing concatenation operations along an axis while performing optional set logic (union or intersection) of the indexes (if any) on the other axes. Note that I say “if any” because there is only a single possible axis of concatenation for Series.

\end{document}
