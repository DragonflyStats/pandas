% Greg Reda

Working with DataFrames

Now that we can get data into a DataFrame, we can finally start working with them. pandas has an abundance of functionality, far too much for me to cover in this introduction. I'd encourage anyone interested in diving deeper into the library to check out its excellent documentation. Or just use Google - there are a lot of Stack Overflow questions and blog posts covering specifics of the library.

We'll be using the MovieLens dataset in many examples going forward. The dataset contains 100,000 ratings made by 943 users on 1,682 movies.


%--------------------------------%
In [31]:
# pass in column names for each CSV
u_cols = ['user_id', 'age', 'sex', 'occupation', 'zip_code']
users = pd.read_csv('ml-100k/u.user', sep='|', names=u_cols)

r_cols = ['user_id', 'movie_id', 'rating', 'unix_timestamp']
ratings = pd.read_csv('ml-100k/u.data', sep='\t', names=r_cols)

# the movies file contains columns indicating the movie's genres
# let's only load the first five columns of the file with usecols
m_cols = ['movie_id', 'title', 'release_date', 'video_release_date', 'imdb_url']
movies = pd.read_csv('ml-100k/u.item', sep='|', names=m_cols, usecols=range(5))
%--------------------------------%
Inspection

pandas has a variety of functions for getting basic information about your DataFrame, the most basic of which is calling your DataFrame by name.
%--------------------------------%
In [32]:
movies
Out[32]:
<class 'pandas.core.frame.DataFrame'>
Int64Index: 1682 entries, 0 to 1681
Data columns (total 5 columns):
movie_id              1682  non-null values
title                 1682  non-null values
release_date          1681  non-null values
video_release_date    0  non-null values
imdb_url              1679  non-null values
dtypes: float64(1), int64(1), object(3)
The output tells a few things about our DataFrame.

It's obviously an instance of a DataFrame.
Each row was assigned an index of 0 to N-1, where N is the number of rows in the DataFrame. pandas will do this by default if an index is not specified. Don't worry, this can be changed later.
There are 1,682 rows (every row must have an index).
Our dataset has five total columns, one of which isn't populated at all (video_release_date) and two that are missing some values (release_date and imdb_url).
The last line displays the datatypes of each column, but not necessarily in the corresponding order to the listed columns. You should use the dtypes method to get the datatype for each column.
In [33]:
movies.dtypes
Out[33]:
movie_id                int64
title                  object
release_date           object
video_release_date    float64
imdb_url               object
dtype: object
DataFrame's also have a describe method, which is great for seeing basic statistics about the dataset's numeric columns. Be careful though, since this will return information on all columns of a numeric datatype.

In [34]:
users.describe()
Out[34]:
user_id	age
count	 943.000000	 943.000000
mean	 472.000000	 34.051962
std	 272.364951	 12.192740
min	 1.000000	 7.000000
25%	 236.500000	 25.000000
50%	 472.000000	 31.000000
75%	 707.500000	 43.000000
max	 943.000000	 73.000000
Notice user_id was included since it's numeric. Since this is an ID value, the stats for it don't really matter.

We can quickly see the average age of our users is just above 34 years old, with the youngest being 7 and the oldest being 73. The median age is 31, with the youngest quartile of users being 25 or younger, and the oldest quartile being at least 43.

You've probably noticed that I've used the head method regularly throughout this post - by default, head displays the first five records of the dataset, while tail displays the last five.

In [35]:
print movies.head()
   movie_id              title release_date  video_release_date  \
0         1   Toy Story (1995)  01-Jan-1995                 NaN   
1         2   GoldenEye (1995)  01-Jan-1995                 NaN   
2         3  Four Rooms (1995)  01-Jan-1995                 NaN   
3         4  Get Shorty (1995)  01-Jan-1995                 NaN   
4         5     Copycat (1995)  01-Jan-1995                 NaN   

                                            imdb_url  
0  http://us.imdb.com/M/title-exact?Toy%20Story%2...  
1  http://us.imdb.com/M/title-exact?GoldenEye%20(...  
2  http://us.imdb.com/M/title-exact?Four%20Rooms%...  
3  http://us.imdb.com/M/title-exact?Get%20Shorty%...  
4  http://us.imdb.com/M/title-exact?Copycat%20(1995)  

In [36]:
print movies.tail(3)
      movie_id                                      title release_date  \
1679      1680                       Sliding Doors (1998)  01-Jan-1998   
1680      1681                        You So Crazy (1994)  01-Jan-1994   
1681      1682  Scream of Stone (Schrei aus Stein) (1991)  08-Mar-1996   

      video_release_date                                           imdb_url  
1679                 NaN      http://us.imdb.com/Title?Sliding+Doors+(1998)  
1680                 NaN  http://us.imdb.com/M/title-exact?You%20So%20Cr...  
1681                 NaN  http://us.imdb.com/M/title-exact?Schrei%20aus%...  

Alternatively, Python's regular slicing syntax works as well.

In [37]:
print movies[20:22]
    movie_id                          title release_date  video_release_date  \
20        21  Muppet Treasure Island (1996)  16-Feb-1996                 NaN   
21        22              Braveheart (1995)  16-Feb-1996                 NaN   

                                             imdb_url  
20  http://us.imdb.com/M/title-exact?Muppet%20Trea...  
21  http://us.imdb.com/M/title-exact?Braveheart%20...  
%--------------------------------%

\subsection{Selecting}

You can think of a DataFrame as a group of Series that share an index (in this case the column headers). This makes it easy to select specific columns.

Selecting a single column from the DataFrame will return a Series object.

\begin{framed}
\begin{verbatim}
users['occupation'].head()
Out[38]:
0    technician
1         other
2        writer
3    technician
4         other
Name: occupation, dtype: object
\begin{verbatim}
\end{framed}

%-----------------------------------------%
To select multiple columns, simply pass a list of column names to the DataFrame, the output of which will be a DataFrame.


\begin{framed}
\begin{verbatim}
print users[['age', 'zip_code']].head()
print '\n'

# can also store in a variable to use later
columns_you_want = ['occupation', 'sex'] 
print users[columns_you_want].head()
   age zip_code
0   24    85711
1   53    94043
2   23    32067
3   24    43537
4   33    15213


   occupation sex
0  technician   M
1       other   F
2      writer   M
3  technician   M
4       other   F
\begin{verbatim}
\end{framed}

%-----------------------------------------%
Row selection can be done multiple ways, but doing so by an individual index or boolean indexing are typically easiest.


\begin{framed}
\begin{verbatim}
# users older than 25
print users[users.age > 25].head(3)
print '\n'

# users aged 40 AND male
print users[(users.age == 40) & (users.sex == 'M')].head(3)
print '\n'

# users younger than 30 OR female
print users[(users.sex == 'F') | (users.age < 30)].head(3)
   user_id  age sex occupation zip_code
1        2   53   F      other    94043
4        5   33   F      other    15213
5        6   42   M  executive    98101


     user_id  age sex  occupation zip_code
18        19   40   M   librarian    02138
82        83   40   M       other    44133
115      116   40   M  healthcare    97232


   user_id  age sex  occupation zip_code
0        1   24   M  technician    85711
1        2   53   F       other    94043
2        3   23   M      writer    32067

Since our index is kind of meaningless right now, let's set it to the userid_ using the set_index method. By default, set_index returns a new DataFrame, so you'll have to specify if you'd like the changes to occur in place.

This has confused me in the past, so look carefully at the code and output below.


\begin{framed}
\begin{verbatim}
print users.set_index('user_id').head()
print '\n'

print users.head()
print "\n^^^ I didn't actually change the DataFrame. ^^^\n"

with_new_index = users.set_index('user_id')
print with_new_index.head()
print "\n^^^ set_index actually returns a new DataFrame. ^^^\n"
         age sex  occupation zip_code
user_id                              
1         24   M  technician    85711
2         53   F       other    94043
3         23   M      writer    32067
4         24   M  technician    43537
5         33   F       other    15213


   user_id  age sex  occupation zip_code
0        1   24   M  technician    85711
1        2   53   F       other    94043
2        3   23   M      writer    32067
3        4   24   M  technician    43537
4        5   33   F       other    15213

^^^ I didn't actually change the DataFrame. ^^^

         age sex  occupation zip_code
user_id                              
1         24   M  technician    85711
2         53   F       other    94043
3         23   M      writer    32067
4         24   M  technician    43537
5         33   F       other    15213

^^^ set_index actually returns a new DataFrame. ^^^


If you want to modify your existing DataFrame, use the inplace parameter.


\begin{framed}
\begin{verbatim}
users.set_index('user_id', inplace=True)
print users.head()
         age sex  occupation zip_code
user_id                              
1         24   M  technician    85711
2         53   F       other    94043
3         23   M      writer    32067
4         24   M  technician    43537
5         33   F       other    15213
\begin{verbatim}
\end{framed}

%-----------------------------------------%
Notice that we've lost the default pandas 0-based index and moved the user_id into its place. We can select rows based on the index using the ix method.


\begin{framed}
\begin{verbatim}
print users.ix[99]
print '\n'
print users.ix[[1, 50, 300]]
age                20
sex                 M
occupation    student
zip_code        63129
Name: 99, dtype: object


     age sex  occupation zip_code
1     24   M  technician    85711
50    21   M      writer    52245
300   26   F  programmer    55106

If we realize later that we liked the old pandas default index, we can just reset_index. The same rules for inplace apply.

In [44]:
users.reset_index(inplace=True)
print users.head()
   user_id  age sex  occupation zip_code
0        1   24   M  technician    85711
1        2   53   F       other    94043
2        3   23   M      writer    32067
3        4   24   M  technician    43537
4        5   33   F       other    15213
\begin{verbatim}
\end{framed}

%-----------------------------------------%
I've found that I can usually get by with boolean indexing and the ix method, but pandas has a whole host of other ways to do selection.

\subsection{Joining}

Throughout an analysis, we'll often need to merge/join datasets as data is typically stored in a relational manner.

Our MovieLens data is a good example of this - a rating requires both a user and a movie, and the datasets are linked together by a key - in this case, the user_id and movie_id. It's possible for a user to be associated with zero or many ratings and movies. Likewise, a movie can be rated zero or many times, by a number of different users.

Like SQL's JOIN clause, pandas.merge allows two DataFrames to be joined on one or more keys. The function provides a series of parameters (on, left_on, right_on, left_index, right_index) allowing you to specify the columns or indexes on which to join.

By default, pandas.merge operates as an inner join, which can be changed using the how parameter.

From the function's docstring:

how : {'left', 'right', 'outer', 'inner'}, default 'inner'

left: use only keys from left frame (SQL: left outer join)
right: use only keys from right frame (SQL: right outer join)
outer: use union of keys from both frames (SQL: full outer join)
inner: use intersection of keys from both frames (SQL: inner join)
Below are some examples of what each look like.


\begin{framed}
\begin{verbatim}
left_frame = pd.DataFrame({'key': range(5), 
                           'left_value': ['a', 'b', 'c', 'd', 'e']})
right_frame = pd.DataFrame({'key': range(2, 7), 
                           'right_value': ['f', 'g', 'h', 'i', 'j']})
print left_frame
print '\n'
print right_frame
   key left_value
0    0          a
1    1          b
2    2          c
3    3          d
4    4          e


   key right_value
0    2           f
1    3           g
2    4           h
3    5           i
4    6           j
\begin{verbatim}
\end{framed}

%-----------------------------------------%
inner join (default)

In [46]:
print pd.merge(left_frame, right_frame, on='key', how='inner')
   key left_value right_value
0    2          c           f
1    3          d           g
2    4          e           h
\begin{verbatim}
\end{framed}

%-----------------------------------------%
We lose values from both frames since certain keys do not match up. The SQL equivalent is:

    SELECT left_frame.key, left_frame.left_value, right_frame.right_value
    FROM left_frame
    INNER JOIN right_frame
        ON left_frame.key = right_frame.key;
Had our key columns not been named the same, we could have used the left_on and right_on parameters to specify which fields to join from each frame.

    pd.merge(left_frame, right_frame, left_on='left_key', right_on='right_key')
Alternatively, if our keys were indexes, we could use the left_index or right_index parameters, which accept a True/False value. You can mix and match columns and indexes like so:

    pd.merge(left_frame, right_frame, left_on='key', right_index=True)
left outer join


\begin{framed}
\begin{verbatim}
print pd.merge(left_frame, right_frame, on='key', how='left')
   key left_value right_value
0    0          a         NaN
1    1          b         NaN
2    2          c           f
3    3          d           g
4    4          e           h

We keep everything from the left frame, pulling in the value from the right frame where the keys match up. The right_value is NULL where keys do not match (NaN).

SQL Equivalent:

SELECT left_frame.key, left_frame.left_value, right_frame.right_value
FROM left_frame
LEFT JOIN right_frame
    ON left_frame.key = right_frame.key;
right outer join

In [48]:
print pd.merge(left_frame, right_frame, on='key', how='right')
   key left_value right_value
0    2          c           f
1    3          d           g
2    4          e           h
3    5        NaN           i
4    6        NaN           j

This time we've kept everything from the right frame with the left_value being NULL where the right frame's key did not find a match.

SQL Equivalent:

SELECT right_frame.key, left_frame.left_value, right_frame.right_value
FROM left_frame
RIGHT JOIN right_frame
    ON left_frame.key = right_frame.key;
full outer join


\begin{framed}
\begin{verbatim}
print pd.merge(left_frame, right_frame, on='key', how='outer')
   key left_value right_value
0    0          a         NaN
1    1          b         NaN
2    2          c           f
3    3          d           g
4    4          e           h
5    5        NaN           i
6    6        NaN           j
\begin{verbatim}
\end{framed}

%-----------------------------------------%
We've kept everything from both frames, regardless of whether or not there was a match on both sides. Where there was not a match, the values corresponding to that key are NULL.

SQL Equivalent (though some databases don't allow FULL JOINs (e.g. MySQL)):

SELECT IFNULL(left_frame.key, right_frame.key) key
        , left_frame.left_value, right_frame.right_value
FROM left_frame
FULL OUTER JOIN right_frame
    ON left_frame.key = right_frame.key;
%--------------------------------%
\newpage
Combining

pandas also provides a way to combine DataFrames along an axis - pandas.concat. While the function is equivalent to SQL's UNION clause, there's a lot more that can be done with it.

pandas.concat takes a list of Series or DataFrames and returns a Series or DataFrame of the concatenated objects. Note that because the function takes list, you can combine many objects at once.


\begin{framed}
\begin{verbatim}
pd.concat([left_frame, right_frame])
Out[50]:
key	left_value	right_value
0	 0	 a	 NaN
1	 1	 b	 NaN
2	 2	 c	 NaN
3	 3	 d	 NaN
4	 4	 e	 NaN
0	 2	 NaN	 f
1	 3	 NaN	 g
2	 4	 NaN	 h
3	 5	 NaN	 i
4	 6	 NaN	 j
\begin{verbatim}
\end{framed}

%-----------------------------------------%
By default, the function will vertically append the objects to one another, combining columns with the same name. We can see above that values not matching up will be NULL.

Additionally, objects can be concatentated side-by-side using the function's axis parameter.


\begin{framed}
\begin{verbatim}
pd.concat([left_frame, right_frame], axis=1)
Out[51]:
key	left_value	key	right_value
0	 0	 a	 2	 f
1	 1	 b	 3	 g
2	 2	 c	 4	 h
3	 3	 d	 5	 i
4	 4	 e	 6	 j
pandas.concat can be used in a variety of ways; however, I've typically only used it to combine Series/DataFrames into one unified object. The documentation has some examples on the ways it can be used.

%-------------------------------------------%
\subsection{Grouping}

Grouping in pandas took some time for me to grasp, but it's pretty awesome once it clicks.

pandas groupby method draws largely from the split-apply-combine strategy for data analysis. If you're not familiar with this methodology, I highly suggest you read up on it. It does a great job of illustrating how to properly think through a data problem, which I feel is more important than any technical skill a data analyst/scientist can possess.

When approaching a data analysis problem, you'll often break it apart into manageable pieces, perform some operations on each of the pieces, and then put everything back together again (this is the gist split-apply-combine strategy). pandas groupby is great for these problems (R users should check out the plyr and dplyr packages).

If you've ever used SQL's GROUP BY or an Excel Pivot Table, you've thought with this mindset, probably without realizing it.

Assume we have a DataFrame and want to get the average for each group - visually, the split-apply-combine method looks like this:

Source: Gratuitously borrowed from Hadley Wickham's Data Science in R slides
Source: Gratuitously borrowed from Hadley Wickham's Data Science in R slides


The City of Chicago is kind enough to publish all city employee salaries to its open data portal. Let's go through some basic groupby examples using this data.


\begin{framed}
\begin{verbatim}
!head -n 3 city-of-chicago-salaries.csv
Name,Position Title,Department,Employee Annual Salary
"AARON,  ELVIA J",WATER RATE TAKER,WATER MGMNT,$85512.00
"AARON,  JEFFERY M",POLICE OFFICER,POLICE,$75372.00
\begin{verbatim}
\end{framed}

%-----------------------------------------%
Since the data contains a dollar sign for each salary, python will treat the field as a series of strings. We can use the converters parameter to change this when reading in the file.

converters : dict. optional

Dict of functions for converting values in certain columns. Keys can either be integers or column labels

\begin{framed}
\begin{verbatim}
headers = ['name', 'title', 'department', 'salary']
chicago = pd.read_csv('city-of-chicago-salaries.csv',
                      header=False,
                      names=headers,
                      converters={'salary': lambda x: float(x.replace('$', ''))})
print chicago.head()
                    name                     title        department  salary
0        AARON,  ELVIA J          WATER RATE TAKER       WATER MGMNT   85512
1      AARON,  JEFFERY M            POLICE OFFICER            POLICE   75372
2    AARON,  KIMBERLEI R  CHIEF CONTRACT EXPEDITER  GENERAL SERVICES   80916
3    ABAD JR,  VICENTE M         CIVIL ENGINEER IV       WATER MGMNT   99648
4  ABBATACOLA,  ROBERT J       ELECTRICAL MECHANIC          AVIATION   89440
\begin{verbatim}
\end{framed}

%-----------------------------------------%
pandas groupby returns a DataFrameGroupBy object which has a variety of methods, many of which are similar to standard SQL aggregate functions.


\begin{framed}
\begin{verbatim}
by_dept = chicago.groupby('department')
print by_dept
<pandas.core.groupby.DataFrameGroupBy object at 0x11118a090>

Calling count returns the total number of NOT NULL values within each column. If we were interested in the total number of records in each group, we could use size.

In [55]:
print by_dept.count().head() # NOT NULL records within each column
print '\n'
print by_dept.size().tail() # total records for each department
                   name  title  department  salary
department                                        
ADMIN HEARNG         42     42          42      42
ANIMAL CONTRL        61     61          61      61
AVIATION           1218   1218        1218    1218
BOARD OF ELECTION   110    110         110     110
BOARD OF ETHICS       9      9           9       9


department
PUBLIC LIBRARY     926
STREETS & SAN     2070
TRANSPORTN        1168
TREASURER           25
WATER MGMNT       1857
dtype: int64
\begin{verbatim}
\end{framed}

%-----------------------------------------%
Summation can be done via sum, averaging by mean, etc. (if it's a SQL function, chances are it exists in pandas). Oh, and there's median too, something not available in most databases.


\begin{framed}
\begin{verbatim}
print by_dept.sum()[20:25] # total salaries of each department
print '\n'
print by_dept.mean()[20:25] # average salary of each department
print '\n'
print by_dept.median()[20:25] # take that, RDBMS!
                       salary
department                   
HUMAN RESOURCES     4850928.0
INSPECTOR GEN       4035150.0
IPRA                7006128.0
LAW                31883920.2
LICENSE APPL COMM     65436.0


                         salary
department                     
HUMAN RESOURCES    71337.176471
INSPECTOR GEN      80703.000000
IPRA               82425.035294
LAW                70853.156000
LICENSE APPL COMM  65436.000000


                   salary
department               
HUMAN RESOURCES     68496
INSPECTOR GEN       76116
IPRA                82524
LAW                 66492
LICENSE APPL COMM   65436
\begin{verbatim}
\end{framed}

%-----------------------------------------%
\end{document}
