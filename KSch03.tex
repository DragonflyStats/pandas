
\documentclass[11pt]{article} % use larger type; default would be 10pt

\usepackage[utf8]{inputenc} 
\usepackage{geometry} % to change the page dimensions
\geometry{a4paper} 
\usepackage{graphicx} 
\usepackage{booktabs} % for much better looking tables
\usepackage{array} % for better arrays (eg matrices) in maths
\usepackage{paralist} % very flexible & customisable lists (eg. enumerate/itemize, etc.)
\usepackage{verbatim} % adds environment for commenting out blocks of text & for better verbatim
\usepackage{subfig} 
\usepackage{framed}
\usepackage{subfiles}
\usepackage{fancyhdr} % This should be set AFTER setting up the page geometry
\pagestyle{fancy} % options: empty , plain , fancy
\renewcommand{\headrulewidth}{0pt} % customise the layout...
\lhead{}\chead{Data Analysis with Python}\rhead{}
\lfoot{}\cfoot{\thepage}\rfoot{}
%--------------------------------------------------------------------------------------------%
\usepackage{sectsty}
\allsectionsfont{\sffamily\mdseries\upshape} 
\usepackage[nottoc,notlof,notlot]{tocbibind} % Put the bibliography in the ToC
\usepackage[titles,subfigure]{tocloft} % Alter the style of the Table of Contents
\renewcommand{\cftsecfont}{\rmfamily\mdseries\upshape}
\renewcommand{\cftsecpagefont}{\rmfamily\mdseries\upshape} % No bold!
%--------------------------------------------------------------------------------------------%

\title{Brief Article}
\author{The Author}
%--------------------------------------------------------------------------------------------%

\begin{document}


% -http://www.kevinsheppard.com/images/0/09/Python_introduction.pdf
Before diving into Python for analyzing data or running Monte Carlos, it is necessary to understand some basic concepts about the core Python data types.

% Unlike domain-specific languages such as MATLAB or R, where the default data type has been chosen for numerical work, Python is a general purpose programming language which is also well suited to data analysis, econometrics and statistics.

 For example,
the basic numeric type in MATLAB is an array (using double precision, which is useful for floating point
mathematics), while the basic numeric data type in Python is a 1-dimensional scalar which may be either
an integer or a double-precision floating point, depending on the formatting of the number when input.

\subsection{3.1 Variable Names}
Variable names can take many forms, although they can only contain numbers, letters (both upper and
lower), and underscores (\_). 

They must begin with a letter or an underscore and are CaSe SeNsItIve.
Additionally, some words are reserved in Python and so cannot be used for variable names (e.g. import or for). For example,

\begin{framed}
\begin{verbatim}
x = 1.0
X = 1.0
X1 = 1.0
X1 = 1.0
x1 = 1.0
dell = 1.0
dellreturns = 1.0
dellReturns = 1.0
_x = 1.0
x_ = 1.0
\end{verbatim}
\end{framed}

are all legal and distinct variable names. Note that names which begin or end with an underscore, while
legal, are not normally used since by convention these convey special meaning.1 Illegal names do not
follow these rules.
% -------------------------------------------------------------------------------------- %
\newpage

\begin{framed}
\begin{verbatim}
>>> x = []
>>> type(x)
builtins.list
>>> x=[1,2,3,4]
>>> x
[1,2,3,4]
# 2-dimensional list (list of lists)
>>> x = [[1,2,3,4], [5,6,7,8]]
>>> x
[[1, 2, 3, 4], [5, 6, 7, 8]]
# Jagged list, not rectangular
>>> x = [[1,2,3,4] , [5,6,7]]
>>> x
[[1, 2, 3, 4], [5, 6, 7]]
# Mixed data types
>>> x = [1,1.0,1+0j,’one’,None,True]
>>> x
[1, 1.0, (1+0j), ’one’, None, True]

\end{verbatim}
\end{framed}

% -------------------------------------------------------------------------------------- %

\end{document}
