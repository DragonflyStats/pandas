\section{Importing and Exporting Data}
%- http://www.kevinsheppard.com/images/0/09/Python_introduction.pdf

\subsection{9.1 Importing Data using pandas}

Pandas is an increasingly important component of the Python scientific stack, and a complete discussion
of its main features is included in Chapter 17. 

All of the data readers in pandas load data into a pandas
DataFrame (see Section 17.1.2), and so these examples all make use of the values property to extract a
NumPy array. 

In practice, the DataFrame is much more useful since it includes useful information such
as column names read from the data source. In addition to the three formats presented here, pandas can
also read json, SQL, html tables or from the clipboard, which is particularly useful for interactive work
since virtually any source that can be copied to the clipboard can be imported.

%===============================================%
\subsection{9.1.1 CSV and other formatted text files}

Comma-separated value (CSV) files can be read using read_csv. When the CSV file contains mixed data,
the default behavior will read the file into an array with an object data type, and so further processing is
usually required to extract the individual series.

\begin{framed}
\begin{verbatim}
>>> from pandas import read_csv
>>> csv_data = read_csv(’FTSE_1984_2012.csv’)
>>> csv_data = csv_data.values
>>> csv_data[:4]
array([[’2012-02-15’, 5899.9, 5923.8, 5880.6, 5892.2, 801550000L, 5892.2],
[’2012-02-14’, 5905.7, 5920.6, 5877.2, 5899.9, 832567200L, 5899.9],
[’2012-02-13’, 5852.4, 5920.1, 5852.4, 5905.7, 643543000L, 5905.7],
[’2012-02-10’, 5895.5, 5895.5, 5839.9, 5852.4, 948790200L, 5852.4]], dtype=object)
>>> open = csv_data[:,1]
\end{verbatim}
\end{framed}
%===============================================%
\end{document}
