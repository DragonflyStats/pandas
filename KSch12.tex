Elements from NumPy arrays can be selected using four methods: scalar selection, slicing, numerical (or
list-of-locations) indexing and logical (or Boolean) indexing.

Numerical indexing uses lists or arrays of locations
to select elements while logical indexing uses arrays containing Boolean values to select elements.


\begin{verbatim}
>>> x = 10 * arange(5.0)
>>> x[[0]] # List with 1 element
array([ 0.])
>>> x[[0,2,1]] # List
array([ 0., 20., 10.])
>>> sel = array([4,2,3,1,4,4]) # Array with repetition
>>> x[sel]
array([ 40., 20., 30., 10., 40., 40.])
>>> sel = array([[4,2],[3,1]]) # 2 by 2 array
>>> x[sel] # Selection has same size as sel
array([[ 40., 20.],
[ 30., 10.]])
>>> sel = array([0.0,1]) # Floating point data
>>> x[sel] # Error
IndexError: arrays used as indices must be of integer (or boolean) type
>>> x[sel.astype(int)] # No error
array([ 10., 20.])
>>> x[0] # Scalar selection, not numerical indexing
1.0

\end{verbatim}

%----------------------%

\begin{verbatim}
>>> x = reshape(arange(10.0), (2,5))
>>> x
array([[ 0., 1., 2., 3., 4.],
[ 5., 6., 7., 8., 9.]])
>>> sel = array([0,1])
>>> x[sel,sel] # 1-dim arrays, no broadcasting
array([ 0., 6.])
>>> x[sel, sel+1]
array([ 1., 7.])
>>> sel_row = array([[0,0],[1,1]])
>>> sel_col = array([[0,1],[0,1]])
>>> x[sel_row,sel_col] # 2 by 2, no broadcasting
array([[ 0., 1.],
[ 5., 6.]])
>>> sel_row = array([[0],[1]])
>>> sel_col = array([[0,1]])
>>> x[sel_row,sel_col] # 2 by 1 and 1 by 2 - difference shapes, broadcasted as 2 by 2
array([[ 0., 1.],
[ 5., 6.]])

\end{verbatim}
%------------------------------------------------------------------------------------------%
\subsection{Mixing Numerical Indexing with Scalar Selection}
NumPy permits using difference types of indexing in the same expression. Mixing numerical indexing
with scalar selection is trivial since any scalar can be broadcast to any array shape.


\begin{framed}
\begin{verbatim}
>>> x = array([[1,2],[3,4]])
>>> sel = x <= 3
>>> indices = nonzero(sel)
>>> indices
(array([0, 0, 1], dtype=int64), array([0, 1, 0], dtype=int64))
\end{verbatim}
\end{framed}

\begin{framed}
\begin{verbatim}
>>> x = randn(3)
>>> x
array([-0.5910316 , 0.51475905, 0.68231135])
>>> argwhere(x<0.6)
array([[0],
[1]], dtype=int64)
>>> argwhere(x<-10.0) # Empty array
array([], shape=(0L, 1L), dtype=int64)
>>> x = randn(3,2)
>>> x
array([[ 0.72945913, 1.2135989 ],
[ 0.74005449, -1.60231553],
[ 0.16862077, 1.0589899 ]])
>>> argwhere(x<0)
array([[1, 1]], dtype=int64)
>>> argwhere(x<1)
array([[0, 0],
[1, 0],
[1, 1],
[2, 0]], dtype=int64)

\end{verbatim}
\end{framed}
%--------------------------------------------------%
\end{document}
