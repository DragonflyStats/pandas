
\documentclass[11pt]{article} % use larger type; default would be 10pt

\usepackage[utf8]{inputenc} 
\usepackage{geometry} % to change the page dimensions
\geometry{a4paper} 
\usepackage{graphicx} 
\usepackage{booktabs} % for much better looking tables
\usepackage{array} % for better arrays (eg matrices) in maths
\usepackage{paralist} % very flexible & customisable lists (eg. enumerate/itemize, etc.)
\usepackage{verbatim} % adds environment for commenting out blocks of text & for better verbatim
\usepackage{subfig} 
\usepackage{framed}
\usepackage{subfiles}
\usepackage{fancyhdr} % This should be set AFTER setting up the page geometry
\pagestyle{fancy} % options: empty , plain , fancy
\renewcommand{\headrulewidth}{0pt} % customise the layout...
\lhead{}\chead{Data Analysis with Python}\rhead{}
\lfoot{}\cfoot{\thepage}\rfoot{}
%--------------------------------------------------------------------------------------------%
\usepackage{sectsty}
\allsectionsfont{\sffamily\mdseries\upshape} 
\usepackage[nottoc,notlof,notlot]{tocbibind} % Put the bibliography in the ToC
\usepackage[titles,subfigure]{tocloft} % Alter the style of the Table of Contents
\renewcommand{\cftsecfont}{\rmfamily\mdseries\upshape}
\renewcommand{\cftsecpagefont}{\rmfamily\mdseries\upshape} % No bold!
%--------------------------------------------------------------------------------------------%

\title{Brief Article}
\author{The Author}
%--------------------------------------------------------------------------------------------%

\begin{document}

\section*{Arrays}
Arrays are the base data type in NumPy, are are arrays in some ways similar to lists since they both contain
collections of elements. The focus of this section is on homogeneous arrays containing numeric data
– that is, an array where all elements have the same numeric type (heterogeneous arrays are covered in
Chapters 16 and 17). 

Additionally, arrays, unlike lists, are always rectangular so that all rows have the same
number of elements.

Arrays are initialized from lists (or tuples) using array. Two-dimensional arrays are initialized using
lists of lists (or tuples of tuples, or lists of tuples, etc.), and higher dimensional arrays can be initialized by
further nesting lists or tuples.

%=======================================%
\subsection{Matrix}
Matrices are essentially a subset of arrays, and behave in a virtually identical manner. The two important
differences are:

\begin{itemize}
\item Matrices always have 2 dimensions
\item Matrices follow the rules of linear algebra for $\ast$
\end{itemize}

\newpage
%=======================================%

\subsection{Accessing Elements of an Array}
Four methods are available for accessing elements contained within an array: scalar selection, slicing,
numerical indexing and logical (or Boolean) indexing. Scalar selection and slicing are the simplest and so
are presented first. Numerical indexing and logical indexing both depends on specialized functions and
so these methods are discussed in Chapter 12

%-------------------------------------------------------------------------%
\subsection{The \texttt{import} function}
Python, by default, only has access to a small number of built-in types and functions. The vast majority of
functions are located in modules, and before a function can be accessed, the module which contains the
function must be imported. 

For example, when using ipython --pylab (or any variants), a large number
of modules are automatically imported, including NumPy and matplotlib. This is style of importing useful
for learning and interactive use, but care is needed to make sure that the correct module is imported when
designing more complex programs.

import can be used in a variety of ways. The simplest is to use from module import * which imports
all functions in module. This method of using import can dangerous since if you use it more than once,
it is possible for functions to be hidden by later imports. A better method is to just import the required
functions. This still places functions at the top level of the namespace, but can be used to avoid conflicts.
\begin{verbatim}
from pylab import log2 # Will import log2 only
from scipy import log10 # Will not import the log2 from SciPy
\end{verbatim}
\end{document}
