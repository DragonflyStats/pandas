\documentclass[KSmainSlides.tex]{subfiles} 
\begin{document} 

%======================================================================== %
\begin{frame}
\frametitle{Logical and Relational Operators}
Logical operators are useful when writing code. Logical operators, when combined
with flow control, allow for complex choices to be compactly expressed.

%--------------------------------%
%11.1
\end{frame}
%======================================================================== %
\begin{frame}
\frametitle{Relational Operators}
\Large

The core logical operators are
\begin{center}
\begin{tabular}{|c|l|l|} \hline
Symbol & Function & Definition \\ \hline
$> $     &   \texttt{greater}   & Greater than \\ \hline
$>=$    &  \texttt{greater\_equal} & Greater than or equal to \\ \hline
$< $      &  \texttt{less} & Less than \\ \hline
$<=$    &  \texttt{less\_equal} & Less than or equal to \\ \hline
$==$    &  \texttt{equal} & Equal to \\ \hline
$!=$      & \texttt{not\_equal} & Not equal to \\ \hline
\end{tabular} 
\end{center}
\end{frame}
%======================================================================== %
\begin{frame}[fragile]
\begin{framed}
\begin{verbatim}
>>> x = array([[1,2],[3,
4]])
>>> x > 0
array([[ True, True],
[False, False]], dtype=bool)
>>> x == 3
array([[False, False],
[ True, False]], dtype=bool)
\end{verbatim}
\end{framed}
\end{frame}
%======================================================================== %
\begin{frame}[fragile]
\begin{framed}
\begin{verbatim}
>>> y = array([1,1])
>>> x < y # y broadcast to be (2,2)
array([[False, False],
[ True, True]], dtype=bool)
>>> z = array([[1,1],[1,
1]])
# Same as broadcast y
>>> x < z
array([[False, False],
[ True, True]], dtype=bool)
\end{verbatim}
\end{framed}
\end{frame}
%======================================================================== %
\begin{frame}[fragile]
\frametitle{Logical Operators}
%-----------------------------------------------%
% 11.2


Logical expressions can be combined using four logical devices,

\begin{tabular}{|c|c|c|c|}
	\hline Keyword (Scalar) & Function & Bitwise & True if . . . \\ \hline
	and & logical\_and & Both & True \\ \hline
	or  & logical\_or & &Either or Both True \\ \hline
	not & logical\_not & ~ & Not True \\ \hline
	& logical\_xor & $\wedge$ & One True and One False \\ \hline
\end{tabular} 
%\begin{verbatim}
% BEGIN TABLE
%Keyword (Scalar) & Function & Bitwise & True if . . . \\ \hline
%and & logical_and & Both & True \\ \hline
%or  & logical_or & Either or Both True \\ \hline
%not & logical_not & ~ & Not True \\ \hline
%& logical_xor & ^ & One True and One False \\ \hline

% END OF TABLE
%\end{verbatim}
\end{frame}

%======================================================================== %
\begin{frame}[fragile]

\noindent There are three versions of all operators except XOR. The keyword version (e.g. and) can only be used
with scalars and so it not useful when working with NumPy. Both the function and bitwise operators
can be used with NumPy arrays, although care is requires when using the bitwise operators.
\end{frame}
%======================================================================== %
\begin{frame}[fragile]
	
\begin{framed}
\begin{verbatim}
>>> x = arange(2.0,4)
>>> y = x >= 0
>>> z = x < 2
>>> logical_and(y, z)
array([False, False, True, True, False, False], dtype=bool)
>>> y & z
array([False, False, True, True, False, False], dtype=bool)
>>> (x > 0) & (x < 2)
array([False, False, True, True, False, False], dtype=bool)
>>>
\end{verbatim}
\end{framed}
\end{frame}
%======================================================================== %
\begin{frame}[fragile]

\begin{framed}
\begin{verbatim}
>>> x > 0 & x < 4 # Error
TypeError: ufunc ’bitwise_and’ not supported for the input types, 
and the inputs could not
be safely coerced to any supported types 
according to the casting rule ’’safe’’

>>> ~(y & z) # Not
array([ True, True, False, False, True, True], dtype=bool)
>>>
\end{verbatim}
\end{framed}
\end{frame}
%======================================================================== %
\begin{frame}[fragile]
	\newpage
%\subsection{\texttt{and}, \texttt{or}, \texttt{not} and \texttt{xor}}
%\begin{framed}
%\begin{verbatim}
%>>> x = arange(2.0,4)
%>>> y = x >= 0
%>>> z = x < 2
%>>> logical_and(y, z)
%array([False, False, True, True, False, False], dtype=bool)
%>>> y & z
%array([False, False, True, True, False, False], dtype=bool)
%>>> (x > 0) & (x < 2)
%array([False, False, True, True, False, False], dtype=bool)
%>>> x > 0 & x < 4 # Error
%TypeError: ufunc ’bitwise_and’ not supported for the input types, and the inputs could not
%be safely coerced to any supported types according to the casting rule ’’safe’’
%>>> ~(y & z) # Not
%array([ True, True, False, False, True, True], dtype=bool)
%\end{verbatim}
%\end{framed}
\end{frame}
%======================================================================== %
\begin{frame}[fragile]
\frametitle{Bitwise operators}
Bitwise operators have high priority – higher than logical comparisons – and so parentheses are requires around
comparisons. 
For example, $(x>1) \& (x<5)$ is a valid statement, while $x>1 \& x<5$, which is evaluated as
$(x>(1 \& x))<5$, produces an error.
\end{frame}
%======================================================================== %
\begin{frame}[fragile]
\frametitle{Bitwise operators}
\begin{framed}
\begin{verbatim}
>>> x = arange(2.0,4)
>>> y = x >= 0
>>> z = x < 2
>>> logical_and(y, z)
array([False, False, True, True, False, False], dtype=bool)
>>> y & z
array([False, False, True, True, False, False], dtype=bool)
>>> (x > 0) & (x < 2)
array([False, False, True, True, False, False], dtype=bool)
\end{verbatim}
\end{framed}
\end{frame}
%======================================================================== %
\begin{frame}[fragile]
	\frametitle{Multiple tests: \texttt{all} and \texttt{any}}
The commands all and any take logical input and are self-descriptive. all returns True if all logical elements
in an array are 1.
%----------------------------%
\begin{itemize}
\item  If \texttt{all} is called without any additional arguments on an array, it returns True if all
elements of the array are logical true and 0 otherwise. 
\item 
any returns logical(True) if any element of an array is True.
\end{itemize}
\end{frame}
%======================================================================== %
\begin{frame}[fragile]
	\frametitle{Multiple tests: \texttt{all} and \texttt{any}}
\begin{itemize}
\item Both \texttt{all }and \texttt{any} can be also be used along a specific dimension using a second argument or the
keyword argument axis to indicate the axis of operation (0 is column-wise and 1 is row-wise). 

\item When used column- or row-wise, the output is an array with one less dimension than the input, where each element
of the output contains the truth value of the operation on a column or row.
\end{itemize}

\end{frame}
%======================================================================== %
\begin{frame}[fragile]
	\frametitle{Multiple tests: \texttt{all} and \texttt{any}}
\begin{framed}
\begin{verbatim}
>>> x = array([[1,2][3,4]])
>>> y = x <= 2
>>> y
array([[ True, True],
[False, False]], dtype=bool)
>>> any(y)
True
>>> any(y,0)
array([[ True, True]], dtype=bool)
>>> any(y,1)
array([[ True],
[False]], dtype=bool)
\end{verbatim}
\end{framed}
\end{frame}
%======================================================================== %
\begin{frame}[fragile]
	\frametitle{allclose}
\begin{itemize}
\item \texttt{allclose} can be used to compare two arrays for near equality.
\item  This type of function is important when
comparing floating point values which may be effectively the same although not identical.
\end{itemize}

%----------------------------%
\end{frame}
%======================================================================== %
\begin{frame}[fragile]
\begin{framed}
\begin{verbatim}
>>> eps = np.finfo(np.float64).eps
>>> eps
2.2204460492503131e16
>>> x = randn(2)
>>> y = x + eps
115
>>> x == y
array([False, False], dtype=bool)
>>> allclose(x,y)
True
\end{verbatim}
\end{framed}
\end{frame}
%======================================================================== %
\begin{frame}[fragile]
The tolerance for being close can be set using keyword arguments either relatively (rtol) or absolutely
(atol).
\end{frame}
%======================================================================== %
\begin{frame}[fragile]
\textbf{array\_equal}\\
\texttt{array\_equal} tests if two arrays have the same shape and elements. It is safer than comparing arrays directly
since comparing arrays which are not broadcastable produces an error.
\textbf{array\_array\_equiv}\\
\texttt{array\_equiv} tests if two arrays are equivalent, even if they do not have the exact same shape. Equivalence
is defined as one array being broadcastable to produce the other.
\end{frame}
%======================================================================== %
\begin{frame}[fragile]
\begin{framed}
\begin{verbatim}>>> x = randn(10,1)
>>> y = tile(x,2)
>>> array_equal(x,y)
False
>>> array_equiv(x,y)
True
\end{verbatim}
\end{framed}
\end{frame}
%======================================================================== %
\begin{frame}[fragile]
%page 116
\frametitle{ is$\ast$}
\begin{itemize}
\item A number of special purpose logical tests are provided to determine if an array has special characteristics.
\item Some operate element-by-element and produce an array of the same dimension as the input while other
produce only scalars. #\item These functions all begin with \texttt{is}.
\end{itemize}

\end{frame}
%======================================================================== %
\begin{frame}[fragile]
\begin{center}
\begin{tabular}{|c|c|c|} \hline
Operator & True if . . . & Method of operation \\ \hline \hline
\texttt{isnan} & 1 if nan & element-by-element \\ \hline
\texttt{isinf} & 1 if inf & element-by-element \\ \hline
\texttt{isfinite} & 1 if not inf and not nan & element-by-element \\ \hline
\texttt{isposfin,isnegfin} & 1 for positive or negative inf & element-by-element \\ \hline
\texttt{isreal} & 1 if not complex valued & element-by-element \\ \hline
\end{tabular}
\end{center}

\end{frame}
%======================================================================== %
\begin{frame}[fragile]
	\begin{center}
		\begin{tabular}{|c|c|c|} \hline
iscomplex &1 if complex valued & element-by-element \\ \hline
isreal & 1 if real valued & element-by-element \\ \hline
is\_string\_like & 1 if argument is a string & scalar \\ \hline
is\_numlike& 1 if is a numeric type & scalar \\ \hline
isscalar & 1 if scalar & scalar \\ \hline
isvector & 1 if input is a vector & scalar \\ \hline
\end{tabular}
\end{center}

\end{frame}
%======================================================================== %


\end{document}
