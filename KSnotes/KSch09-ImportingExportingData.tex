\documentclass[KSmain.tex]{subfiles} 
\begin{document} 
%===============================================%

%	9.1 Importing Data using pandas	
%		9.1.1 CSV and other formatted text files
%		9.1.2 Excel files
%		9.1.3 STATA files
%	9.2 Importing Data without pandas	NOT USING
%		9.2.1 CSV and other formatted text files
%		9.2.2 Excel Files
%		9.2.3 MATLAB Data Files (.mat)
%		9.2.4 Reading Complex Files
%	9.3 Saving or Exporting Data using pandas	
%	9.4 Saving or Exporting Data without pandas	 NOT USING
%		9.4.1 Writing MATLAB Data Files (.mat)
%		9.4.2 Exporting Data to Text Files

%===============================================%

\section{Importing and Exporting Data}
%- http://www.kevinsheppard.com/images/0/09/Python_introduction.pdf

% 9.1
\subsection{Importing Data using pandas}

%Pandas is an increasingly important component of the Python scientific stack, and a complete discussion
% of its main features is included in Chapter 17. 

All of the data reading functions in pandas load data into a pandas
DataFrame, and so these examples all make use of the values property to extract a
NumPy array. 

In practice, the DataFrame is much more useful since it includes useful information such
as column names read from the data source. 
In addition to the three main formats, pandas can
also read json, SQL, html tables or from the clipboard, which is particularly useful for interactive work
since virtually any source that can be copied to the clipboard can be imported.

%===============================================%
% 9.1.1 
\subsection{CSV and other formatted text files}

Comma-separated value (CSV) files can be read using \texttt{read\_csv}. When the CSV file contains \textit{mixed data},
the default behavior will read the file into an array with an object data type, and so further processing is
usually required to extract the individual series.

\begin{framed}
\begin{verbatim}
>>> from pandas import read_csv
>>> csv_data = read_csv(’FTSE_1984_2012.csv’)
>>> csv_data = csv_data.values
>>> csv_data[:4]
array([[’2012-02-15’, 5899.9, 5923.8, 5880.6, 
 5892.2, 801550000L, 5892.2],
[’2012-02-14’, 5905.7, 5920.6, 5877.2, 5899.9, 832567200L, 5899.9],
[’2012-02-13’, 5852.4, 5920.1, 5852.4, 5905.7, 643543000L, 5905.7],
[’2012-02-10’, 5895.5, 5895.5, 5839.9, 5852.4, 948790200L, 5852.4]], 
dtype=object)
>>> open = csv_data[:,1]
\end{verbatim}
\end{framed}

When the entire file is numeric, the data will be stored as a homogeneous array using one of the numeric
data types, typically float64. In this example, the first column contains Excel dates as numbers, which are
the number of days past January 1, 1900.
\begin{framed}
	\begin{verbatim}
>>> csv_data = read_csv(’FTSE_1984_2012_numeric.csv’)
>>> csv_data = csv_data.values
>>> csv_data[:4,:2]
array([[ 40954. , 5899.9],
[ 40953. , 5905.7],
[ 40952. , 5852.4],
[ 40949. , 5895.5]])
\end{verbatim}
\end{framed}
\subsubsection{Excel files}
Excel files, both 97/2003 (xls) and 2007/10/13 (xlsx), can be imported using \texttt{read\_excel}. Two inputs are
required to use \texttt{read\_excel}, the filename and the sheet name containing the data. In this example, pandas
makes use of the information in the Excel workbook that the first column contains dates and converts
these to datetimes. Like the mixed CSV data, the array returned has object data type.
\begin{framed}
\begin{verbatim}
>>> from pandas import read_excel
>>> excel_data = read_excel(’FTSE_1984_2012.xls’,’FTSE_1984_2012’)
>>> excel_data = excel_data.values
>>> excel_data[:4,:2]
array([[datetime.datetime(2012, 2, 15, 0, 0), 5899.9],
[datetime.datetime(2012, 2, 14, 0, 0), 5905.7],
[datetime.datetime(2012, 2, 13, 0, 0), 5852.4],
[datetime.datetime(2012, 2, 10, 0, 0), 5895.5]], dtype=object)
>>> open = excel_data[:,1]
\end{verbatim}
\end{framed}
%===============================================%
%---------------------------------------%
\newpage
\end{document}

