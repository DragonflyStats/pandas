% 12.1 Numerical Indexing - Shortened Version
% - 12.1.1 Mixing Numerical Indexing with Scalar Selection
% - 12.1.2 Mixing Numerical Indexing with Slicing
% - 12.1.3 Linear Numerical Indexing using flat
% - 12.1.4 Mixing Numerical Indexing with Slicing and Scalar Selection
% 12.2 Logical Indexing - Shortened Version
% - 12.2.1 Mixing Logical Indexing with Scalar Selection
% - 12.2.3 Mixing Logical Indexing with Numerical Indexing
% - 12.2.2 Mixing Logical Indexing with Slicing
% - 12.2.4 Logical Indexing Functions

% 12.3 Performance Considerations NOT USED
% 12.4 Assignment with Broadcasting NOT USED
\documentclass[KSmain.tex]{subfiles} 
\begin{document} 

\section{Data Wrangling}
Elements from NumPy arrays can be selected using four methods: scalar selection, slicing, numerical (or
list-of-locations) indexing and logical (or Boolean) indexing.


%12.1
\subsection{Numerical Indexing}
Numerical indexing uses lists or arrays of locations
to select elements while logical indexing uses arrays containing Boolean values to select elements.

Numerical indexing, also called list-of-location indexing, is an alternative to slice notation. The fundamental
idea underlying numerical indexing is to use coordinates to select elements, which is similar to
the underlying idea behind slicing.

A numerical index can be either a list or a NumPy array and must contain integer data.
\begin{framed}
\begin{verbatim}
>>> x = 10 * arange(5.0)
>>> x[[0]] # List with 1 element
array([ 0.])
>>> x[[0,2,1]] # List
array([ 0., 20., 10.])
>>> sel = array([4,2,3,1,4,4]) # Array with repetition
>>>
>>> x[sel]
array([ 40., 20., 30., 10., 40., 40.])
>>> sel = array([[4,2],[3,1]]) # 2 by 2 array
>>> x[sel] # Selection has same size as sel
array([[ 40., 20.],
[ 30., 10.]])
>>> sel = array([0.0,1]) # Floating point data
>>>
>>> x[sel] # Error
IndexError: arrays used as indices must be of integer (or boolean) type
>>> x[sel.astype(int)] # No error
array([ 10., 20.])
>>> x[0] # Scalar selection, not numerical indexing
1.0

\end{verbatim}
\end{framed}

%----------------------%

\begin{framed}
\begin{verbatim}
>>> x = reshape(arange(10.0), (2,5))
>>> x
array([[ 0., 1., 2., 3., 4.],
[ 5., 6., 7., 8., 9.]])
>>> sel = array([0,1])
>>> x[sel,sel] # 1-dim arrays, no broadcasting
array([ 0., 6.])
>>> x[sel, sel+1]
array([ 1., 7.])
>>> sel_row = array([[0,0],[1,1]])
>>> sel_col = array([[0,1],[0,1]])
>>> x[sel_row,sel_col] # 2 by 2, no broadcasting
array([[ 0., 1.],
[ 5., 6.]])
>>>
>>> sel_row = array([[0],[1]])
>>> sel_col = array([[0,1]])
>>> # 2 by 1 and 1 by 2 - difference shapes, broadcasted as 2 by 2
>>> x[sel_row,sel_col] 
array([[ 0., 1.],
[ 5., 6.]])
\end{verbatim}
\end{framed}
%------------------------------------------------------------------------------------------%
%12.1.1
\subsubsection*{Mixing Numerical Indexing with Scalar Selection}
NumPy permits using difference types of indexing in the same expression. Mixing numerical indexing
with scalar selection is trivial since any scalar can be broadcast to any array shape.


\begin{framed}
\begin{verbatim}
>>> x = array([[1,2],[3,4]])
>>> sel = x <= 3
>>> indices = nonzero(sel)
>>> indices
(array([0, 0, 1], dtype=int64), array([0, 1, 0], dtype=int64))
\end{verbatim}
\end{framed}

%12.1.2 Mixing Numerical Indexing with Slicing
\subsubsection*{Mixing Numerical Indexing with Slicing}

Mixing numerical indexing and slicing allow for entire rows or columns to be selected.

\begin{framed}
	\begin{verbatim}
	>>> x[:,[1]]
	array([[ 2.],
	[ 7.]])
	>>> x[[1],:]
	array([[ 6., 7., 8., 9., 10.]])
	
	\end{verbatim}
\end{framed}

Note that the mixed numerical indexing and slicing uses a list ([1]) so that it is not a scalar. This is important
since using a scalar will result in dimension reduction.


\begin{framed}
	\begin{verbatim}
>>> x[:,1] # 1dimensional
array([ 2., 7.])
	\end{verbatim}
\end{framed}

Numerical indexing and slicing can be mixed in more than 2-dimensions, although some care is required.
In the simplest case where only one numerical index is used which is 1-dimensional, then the selection is
equivalent to calling \texttt{ix\_} where the slice a:b:s is replaced with arange(a,b,s).



\begin{framed}
	\begin{verbatim}
	>>> x = reshape(arange(3**3), (3,3,3)) # 3d
	array
	>>> sel1 = x[::2,[1,0],:1]
	>>> sel2 = x[ix_(arange(0,3,2),[1,0],arange(0,1))]
	>>> sel1.shape
	(2L, 2L, 1L)
	>>> sel2.shape
	(2L, 2L, 1L)
	>>> amax(abs(sel1sel2))
	0
	\end{verbatim}
\end{framed}

%12.1.3
\subsubsection*{Linear Numerical Indexing using flat}
Like slicing, numerical indexing can be combined with flat to select elements from an array using the
row-major ordering of the array. The behavior of numerical indexing with flat is identical to that of using
numerical indexing on a flattened version of the underlying array.
\begin{framed}
	\begin{verbatim}
	>>> x.flat[[3,4,9]]
array([ 4., 5., 10.])
>>> x.flat[[[3,4,9],[1,5,3]]]
array([[ 4., 5., 10.],
[ 2., 6., 4.]])
	\end{verbatim}
\end{framed}

%--------------------------------------------------%
%12.2 Logical Indexing
\newpage
\subsection{Logical Indexing}
Logical indexing differs fromslicing and numeric indexing by using logical indices to select elements, rows
or columns. Logical indices act as light switches and are either “on” (True) or “off” (False). Pure logical
indexing uses a logical indexing array with the same size as the array being used for selection and always
returns a 1-dimensional array.
\begin{framed}
\begin{verbatim}
>>> x = arange(3,3)
>>> x < 0
array([ True, True, True, False, False, False], dtype=bool)
>>> x[x < 0]
array([3,
2,
1])
>>> x[abs(x) >= 2]
array([3,
2,
2])
>>> x = reshape(arange(8,
8), (4,4))
>>> x[x < 0]
array([8,
7,
6,
5,
4,
3,
2,
1])
\end{verbatim}
\end{framed}
It is tempting to use two 1-dimensional logical arrays to act as row and column masks on a 2-dimensional
array. This does not work, and it is necessary to use \texttt{ix\_} if interested in this type of indexing.
\begin{framed}
\begin{verbatim}
>>> x = reshape(arange(8,8),(
4,4))
>>> cols = any(x < 6,
0)
>>> rows = any(x < 0, 1)
>>> cols
array([ True, True, False, False], dtype=bool
>>> rows
array([ True, True, False, False], dtype=bool)
>>> x[cols,rows] # Not upper 2 by 2
array([8,
3])
>>> x[ix_(cols,rows)] # Upper 2 by 2
array([[8,
7],
[4,
3]])
\end{verbatim}
\end{framed}
The difference between the final 2 commands is due to how logical indexing operates when more than
logical array is used. When using 2 or more logical indices, they are first transformed to numerical indices
using nonzero which returns the locations of the non-zero elements (which correspond to the True
elements of a Boolean array).
\begin{framed}
\begin{verbatim}
>>> cols.nonzero()
(array([0, 1], dtype=int64),)
>>> rows.nonzero()
(array([0, 1], dtype=int64),)
\end{verbatim}
\end{framed}
The corresponding numerical index arrays have compatible sizes – both are 2-element, 1-dimensional
arrays – and so numeric selection is possible. Attempting to use two logical index arrays which have
non-broadcastable dimensions produces the same error as using two numerical index arrays with nonbroadcastable
sizes.
\begin{framed}
\begin{verbatim}
>>> cols = any(x < 6,
0)
>>> rows = any(x < 4, 1)
>>> rows
array([ True, True, True, False], dtype=bool)
>>> x[cols,rows] # Error
ValueError: shape mismatch: objects cannot be broadcast to a single shape
\end{verbatim}
\end{framed}
\subsubsection*{\texttt{argwhere}}
\texttt{argwhere} returns an array containing the locations of elements where a logical condition is True. It is the
same as \texttt{transpose(nonzero(x))}

\begin{framed}
	\begin{verbatim}
	>>> x = randn(3)
	>>> x
	array([-0.5910316 , 0.51475905, 0.68231135])
	>>> argwhere(x<0.6)
	array([[0],
	[1]], dtype=int64)
	>>> argwhere(x<-10.0) # Empty array
	array([], shape=(0L, 1L), dtype=int64)
	>>>
	>>> x = randn(3,2)
	>>> x
	array([[ 0.72945913, 1.2135989 ],
	[ 0.74005449, -1.60231553],
	[ 0.16862077, 1.0589899 ]])
	>>>
	>>> argwhere(x<0)
	array([[1, 1]], dtype=int64)
	>>>
	>>> argwhere(x<1)
	array([[0, 0],
	[1, 0],
	[1, 1],
	[2, 0]], dtype=int64)
	
	\end{verbatim}
\end{framed}
\end{document}
