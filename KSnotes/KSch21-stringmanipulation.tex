Chapter 21
String Manipulation
Strings are usually less interesting than numerical values in econometrics and statistics. There are, however,
some important uses for strings:
• Reading complex data formats
• Outputting formatted results to screen or file
Recall that strings are sliceable, but unlike arrays, are immutable, and so it is not possible to replace part
of a string.
21.1 String Building
21.1.1 Adding Strings (+)
Strings are concatenated using +.
>>> a = ’Python is’
>>> b = ’a rewarding language.’
>>> a + ’ ’ + b
’Python is a rewarding language.’
While + is a simple method to join strings, the modern method is to use join. join is a string method
which joins a list of strings (the input) using the object calling the string as the separator.
>>> a = ’Python is’
>>> b = ’a rewarding language.’
>>> ’ ’.join([a,b])
’Python is a rewarding language.’
Alternatively, the same output may be constructed using an empty string ’’.
>>> a = ’Python is’
>>> b = ’a rewarding language.’
>>> ’’.join([a,’ ’,b])
’Python is a rewarding language.’
join is also useful for producing comma separated lists.
>>> words = [’Python’,’is’,’a’,’rewarding’,’language’]
>>> ’,’.join(words)
’Python,is,a,rewarding,language’
21.1.2 Multiplying Strings (*)
Strings, like lists, can be repeated using *.
>>> a = ’Python is ’
>>> 2*a
’Python is Python is ’
21.1.3 Using cStringIO
While adding strings using + or join is extremely simple, concatenation is slow for large strings. The module
cStringIO provides an optimized class for performing string operations, including buffering strings
for fast string building. This example shows how write(string) fills a StringIO buffer. Before reading the
contents seek(0) is called to return to cursor to the beginning of the buffer, and then read() returns the
entire string from the buffer.
>>> import cStringIO
>>> sio = cStringIO.StringIO()
>>> for i in xrange(10000):
... sio.write(’cStringIO is faster than +! ’)
>>> sio.seek(0)
>>> sio.read()
Note that this example is trivial since * could have been used instead.
21.2 String Functions
21.2.1 split and rsplit
split splits a string into a list based on a character, for example a comma. An optional third argument
maxsplit can be used to limit the number of outputs in the list. rsplit works identically to split, only
scanning from the end of the string – split and rsplit only differ when maxsplit is used.
>>> s = ’Python is a rewarding language.’
>>> s.split(’ ’)
[’Python’, ’is’, ’a’, ’rewarding’, ’language.’]
>>> s.split(’ ’,3)
[’Python’, ’is’, ’a’, ’rewarding language.’]
>>> s.rsplit(’ ’,3)
[’Python is’, ’a’, ’rewarding’, ’language.’]

21.2.2 join
join concatenates a list or tuple of strings, using an optional argument sep which specified a separator
(default is space).
>>> import string
>>> a = ’Python is’
>>> b = ’a rewarding language.’
>>> string.join((a,b))
’Python is a rewarding language.’
>>> string.join((a,b),’:’)
’Python is:a rewarding language.’
>>> ’ ’.join((a,b)) # Method version
’Python is a rewarding language.’
21.2.3 strip, lstrip, and rstrip
strip removes leading and trailing whitespace from a string. An optional input char removes leading
and trailing occurrences of the input value (instead of space). lstrip and rstrip work identically, only
stripping from the left and right, respectively.
>>> s = ’ Python is a rewarding language. ’
>>> s=s.strip()
’Python is a rewarding language.’
>>> s.strip(’P’)
’ython is a rewarding language.’
21.2.4 find and rfind
find locates the lowest index of a substring in a string and returns -1 if not found. Optional arguments
limit the range of the search, and s.find(’i’,10,20) is identical to s[10:20].find(’i’). rfind works
identically, only returning the highest index of the substring.
>>> s = ’Python is a rewarding language.’
>>> s.find(’i’)
7
>>> s.find(’i’,10,20)
18
>>> s.rfind(’i’)
18
find and rfind are commonly used in flow control.
>>> words = [’apple’,’banana’,’cherry’,’date’]
>>> words_with_a = []
>>> for word in words:
... if word.find(’a’)>=0:
... words_with_a.append(word)
>>> words_with_a
[’apple’, ’banana’, ’date’]
21.2.5 index and rindex
index returns the lowest index of a substring, and is identical to find except that an error is raised if the
substring does not exist. As a result, index is only safe to use in a try . . . except block.
>>> s = ’Python is a rewarding language.’
>>> s.index(’i’)
7
>>> s.index(’q’) # Error
ValueError: substring not found
21.2.6 count
count counts the number of occurrences of a substring, and takes optional arguments to limit the search
range.
>>> s = ’Python is a rewarding language.’
>>> s.count(’i’)
2
>>> s.count(’i’, 10, 20)
1
21.2.7 lower and upper
lower and upper convert strings to lower and upper case, respectively. They are useful to remove case
when comparing strings.
>>> s = ’Python is a rewarding language.’
>>> s.upper()
’PYTHON IS A REWARDING LANGUAGE.’
>>> s.lower()
’python is a rewarding language.’
21.2.8 ljust, rjust and center
ljust, rjust and center left justify, right justify and center, respectively, a string while expanding its size
to a given length. If the desired length is smaller than the string, the unchanged string is returned.
>>> s = ’Python is a rewarding language.’
>>> s.ljust(40)
’Python is a rewarding language. ’
>>> s.rjust(40)
’ Python is a rewarding language.’
>>> s.center(40)
’ Python is a rewarding language. ’
21.2.9 replace
replace replaces a substring with an alternative string, which can have different size. An optional argument
limits the number of replacement.
>>> s = ’Python is a rewarding language.’
>>> s.replace(’g’,’Q’)
’Python is a rewardinQ lanQuaQe.’
>>> s.replace(’is’,’Q’)
’Python Q a rewarding language.’
>>> s.replace(’g’,’Q’,2)
’Python is a rewardinQ lanQuage.’
21.2.10 textwrap.wrap
The module textwrap contains a function wrap which reformats a long string into a fixed width paragraph
stored line-by-line in a list. An optional argument changes the width of the output paragraph form the
default of 70 characters.
>>> import textwrap
>>> s = ’Python is a rewarding language. ’
>>> s = 10*s
>>> textwrap.wrap(s)
[’Python is a rewarding language. Python is a rewarding language. Python’,
’is a rewarding language. Python is a rewarding language. Python is a’,
’rewarding language. Python is a rewarding language. Python is a’,
’rewarding language. Python is a rewarding language. Python is a’,
’rewarding language. Python is a rewarding language.’]
>>> textwrap.wrap(s,50)
[’Python is a rewarding language. Python is a’,
’rewarding language. Python is a rewarding’,
’language. Python is a rewarding language. Python’,
’is a rewarding language. Python is a rewarding’,
’language. Python is a rewarding language. Python’,
’is a rewarding language. Python is a rewarding’,
’language. Python is a rewarding language.’]

21.3 Formatting Numbers
Formatting numbers when converting to a string allows for automatic generation of tables and well formatted
screen output. Numbers are formatted using the format function, which is used in conjunction
with a format specifier. For example, consider these examples which format .
>>> pi
3.141592653589793
>>> ’{:12.5f}’.format(pi)
’ 3.14159’
>>> ’{:12.5g}’.format(pi)
’ 3.1416’
>>> ’{:12.5e}’.format(pi)
’ 3.14159e+00’
These all provide alternative formats and the difference is determined by the letter in the format string.
The generic formof a format string is {n:f a swc .p t } or {n:f a swcmt }. To understand the the various
choices, consider the output produced by the basic output string ’{0:}’
>>> ’{0:}’.format(pi)
’3.14159265359’
• n is a number 0,1,. . . indicating which value to take from the format function
>>> ’{0:}, {1:} and {2:} are all related to pi’.format(pi,pi+1,2*pi)
’3.14159265359, 4.14159265359 and 6.28318530718 are all related to pi’
>>> ’{2:}, {0:} and {1:} reorder the output.’.format(pi,pi+1,2*pi)
’6.28318530718, 3.14159265359 and 4.14159265359 reorder the output.
f a are fill and alignment characters, typically a 2 character string. Fill may be any character except
}, although space is the most common choice. Alignment can < (left) ,> (right), ^ (center) or = (pad
to the right of the sign). Simple left 0-fills can omit the alignment character so that f a = 0.
>>> ’{0:0<20}’.format(pi) # Left, 0 padding, precion 20
’3.141592653590000000’
>>> ’{0:0>20}’.format(pi) # Right, 0 padding, precion 20
’00000003.14159265359’
>>> ’{0:0^20}’.format(pi) # Center, 0 padding, precion 20
’0003.141592653590000’
>>> ’{0: >20}’.format(pi) # Right, space padding, precion 20
’ 3.14159265359’
>>> ’{0:$^20}’.format(pi) # Center, dollar sign padding, precion 20
’$$$3.14159265359$$$$’
