\documentclass[KSmain.tex]{subfiles} 
\begin{document} 

%=====================
% KS chapter Chapter 18

% 18.1 Functions
%  18.1.2 Default Values
% 18.4 Anonymous Function
% 18.8 Python Coding Conventions

\section{Custom Function and Modules}
%------------------------------------------------%
Python supports a wide range of programming styles including procedural (imperative), object oriented
and functional. While object oriented programming and functional programming are powerful programming
paradigms, especially in large, complex software, procedural is often both easier to understand and a
direct representation of a mathematical formula. The basic idea of procedural programming is to produce
a function or set of function (generically) of the form:
\[ y = f (x ).\]
That is, the functions take one or more inputs and produce one or more outputs.

%------------------------------------------------%
% 18.1 Functions
\subsection{Functions}
Python functions are very simple to declare and can occur in the same file as the main program or a standalone
file. Functions are declared using the \texttt{def} keyword, and the value produced is returned using the
\texttt{return} keyword. Consider a simple function which returns the square of the input, $y = x^2$.
\begin{verbatim}
from __future__ import print_function, division

def square(x):
return x**2
# Call the function
x = 2
y = square(x)
print(x,y)
\end{verbatim}
%------------------------------------------------%
In this example, the same Python file contains the main program– the final 3 lines – aswell as the function.
More complex function can be crafted with multiple inputs.
\begin{verbatim}
from __future__ import print_function, division
def l2distance(x,y):
return (xy)**
2
# Call the function
x = 3
y = 10
z = l2distance(x,y)
print(x,y,z)
\end{verbatim}
Function can also be defined using NumPy arrays and matrices.
\begin{verbatim}
from __future__ import print_function, division
import numpy as np
def l2_norm(x,y):
d = x y
return np.sqrt(np.dot(d,d))
# Call the function
x = np.random.randn(10)
y = np.random.randn(10)
z = l2_norm(x,y)
print(xy)
print("The L2 distance is ",z)
\end{verbatim}
When multiple outputs are returned but only a single variable is available for assignment, all outputs are
returned in a tuple. Alternatively, the outputs can be directly assigned when the function is called with
the same number of variables as outputs.
\begin{verbatim}
from __future__ import print_function, division
import numpy as np
def l1_l2_norm(x,y):
d = x y
return sum(np.abs(d)),np.sqrt(np.dot(d,d))
# Call the function
x = np.random.randn(10)
y = np.random.randn(10)
# Using 1 output returns a tuple
z = l1_l2_norm(x,y)
print(xy)
print("The L1 distance is ",z[0])
print("The L2 distance is ",z[1])
# Using 2 output returns the values
l1,l2 = l1_l2_norm(x,y)
print("The L1 distance is ",l1)
print("The L2 distance is ",l2)
\end{verbatim}
All of these functions have been placed in the same file as the main program. 
%Placing functions in modules allows for reuse in multiple programs, and will be discussed later on.

%----------------------------%
% 18.1.1
%----------------------------%
\subsubsection{Keyword Arguments}
All input variables in functions are automatically keyword arguments, so that the function can be accessed
either by placing the inputs in the order they appear in the function (positional arguments), or by calling
the input by their name using keyword=value.
\begin{framed}
	\begin{verbatim}
from __future__ import print_function, division
import numpy as np
def lp_norm(x,y,p):
d = x y
return sum(abs(d)**p)**(1/p)
# Call the function
x = np.random.randn(10)
y = np.random.randn(10)
z1 = lp_norm(x,y,2)
z2 = lp_norm(p=2,x=x,y=y)
print("The Lp distances are ",z1,z2)
\end{verbatim}
\end{framed}

Because variable names are automatically keywords, it is important to use meaningful variable names
when possible, rather than generic variables such as a, b, c or x, y and z. In some cases, x may be a reasonable
default, but in the previous example which computed the Lp norm, calling the third input z would
be bad idea.
%----------------------------%
% 18.1.2 Default Values
%----------------------------%
\subsubsection{Default Values}
Default values are set in the function declaration using the syntax input=default.
\begin{framed}
\begin{verbatim}
from __future__ import print_function, division
import numpy as np
def lp_norm(x,y,p = 2):
d = x y
return sum(abs(d)**p)**(1/p)
# Call the function
x = np.random.randn(10)
y = np.random.randn(10)
# Inputs with default values can be ignored
l2 = lp_norm(x,y)
l1 = lp_norm(x,y,1)
print("The l1 and l2 distances are ",l1,l2)
print("Is the default value overridden?", sum(abs(xy))==
l1)
\end{verbatim}
\end{framed}

Default values should not normally be mutable (e.g. lists or arrays) since they are only initialized the first
time the function is called. Subsequent calls will use the same value, which means that the default value
could change every time the function is called.

\begin{framed}
\begin{verbatim}

from __future__ import print_function, division
import numpy as np
def bad_function(x = zeros(1)):
print(x)
x[0] = np.random.randn(1)
# Call the function
bad_function()
bad_function()
bad_function()
\end{verbatim}
\end{framed}
Each call to \texttt{bad\_function} shows that x has a different value – despite the default being 0. The solution
to this problem is to initialize mutable objects to None, and then the use an if to check and initialize only
if the value is None. Note that tests for None use the is keyword rather the testing for equality using ==.
\begin{framed}
\begin{verbatim}
from __future__ import print_function, division
import numpy as np
def good_function(x = None):
if x is None:
x = zeros(1)
print(x)
x[0] = np.random.randn(1)
# Call the function
good_function()
good_function()
\end{verbatim}
\end{framed}
%----------------------------%
\subsection{Variable Number of Inputs}
Most function written as an “end user” have an known (ex ante) number of inputs. However, functions
which evaluate other functions often must accept variable numbers of input. Variable inputs can be handled
using the *args (arguments) or **kwargs (keyword arguments) syntax. The *args syntax will generate
tuple a containing all inputs past the required input list. For example, consider extending the Lp function
so that it can accept a set of p values as extra inputs (Note: in practice it would make more sense to accept
an array for p).
\begin{framed}
\begin{verbatim}
from __future__ import print_function, division
import numpy as np
def lp_norm(x,y,p = 2, *args):
d = x-y
print(’The L’ + str(p) + ’ distance is :’, sum(abs(d)**p)**(1/p))
out = [sum(abs(d)**p)**(1/p)]
print(’Number of *args:’, len(args))
for p in args:
print(’The L’ + str(p) + ’ distance is :’, sum(abs(d)**p)**(1/p))
out.append(sum(abs(d)**p)**(1/p))
return tuple(out)
# Call the function
x = np.random.randn(10)
y = np.random.randn(10)
# x & y are required inputs and so are not in *args
lp = lp_norm(x,y)
# Function takes 3 inputs, so no *args
lp = lp_norm(x,y,1)
# Inputs with default values can be ignored
lp = lp_norm(x,y,1,2,3,4,1.5,2.5,0.5)
\end{verbatim}
\end{framed}
The alternative syntax, **kwargs, generates a dictionary with all keyword inputs which are not in the
function signature. One reason for using **kwargs is to allow a long list of optional inputs without having
to have an excessively long function definition. This is how this input mechanism operates in many
matplotlib functions such as plot.
\begin{framed}
	\begin{verbatim}
from __future__ import print_function, division
import numpy as np
def lp_norm(x,y,p = 2, **kwargs):
d = x y
print(’Number of *kwargs:’, len(kwargs))
for key in kwargs:
print(’Key :’, key, ’ Value:’, kwargs[key])
return sum(abs(d)**p)
# Call the function
x = np.random.randn(10)
y = np.random.randn(10)
# Inputs with default values can be ignored
lp = lp_norm(x,y,kword1=1,kword2=3.2)
# The p keyword is in the function def, so not in **kwargs
lp = lp_norm(x,y,kword1=1,kword2=3.2,p=0)
\end{verbatim}
\end{framed}


\newpage
% 18.4 Anonymous Functions
\subsection{Anonymous Functions}

Python support anonymous functions using the keyword lambda. Anonymous functions are usually encountered
when another function expects a function as an input and a simple function will suffice. Anonymous
function take the generic formlambda a,b,c,. . .:code using a,b,c. The key elements are the keyword
lambda, a list of comma separated inputs, a colon between the inputs and the actual function code. For
example lambda x,y:x+y would return the sum of the variables x and y.

Anonymous functions are simple but useful. For example, when lists containing other lists it isn’t
directly possible to sort on an arbitrary element of the nested list. Anonymous functions allow sorting
through the keyword argument key by returning the element Python should use to sort. In this example,
a direct call to \texttt{sort()} will sort on the first element (first name). Using the anonymous function
lambda x:x[1] to returnthe second element of the tuple allows for sorting on the lastname. \texttt{lambda x:x[2]}
would allow for sorting on the University.
\begin{framed}
	\begin{verbatim}
>>> nested = [(’John’,’Doe’,’Oxford’),\
... (’Jane’,’Dearing’,’Cambridge’),\
... (’Jerry’,’Dawn’,’Harvard’)]
>>> nested.sort()
>>> nested
[(’Jane’, ’Dearing’, ’Cambridge’),
(’Jerry’, ’Dawn’, ’Harvard’),
(’John’, ’Doe’, ’Oxford’)]
>>> nested.sort(key=lambda x:x[1])
>>> nested
[(’Jerry’, ’Dawn’, ’Harvard’),
(’Jane’, ’Dearing’, ’Cambridge’),
(’John’, ’Doe’, ’Oxford’)]
\end{verbatim}
\end{framed}
%----------------------------%
% 18.8 Python Coding Conventions
\subsection{Python Coding Conventions}
There are a number of common practices which can be adopted to produce Python code which looks
more like code found in other modules:
\begin{enumerate}
\item Use 4 spaces to indent blocks – avoid using tab, except when an editor automatically converts tabs
to 4 spaces
\item Avoid more than 4 levels of nesting, if possible
\item Limit lines to 79 characters. The \ symbol can be used to break long lines
219
\item Use two blank lines to separate functions, and one to separate logical sections in a function.
\item Use ASCII mode in text editors, not UTF-8
\item One module per import line
\item Avoid from module import * (for any module). Use either from module import func1, func2 or
import module as shortname.
\item Follow the NumPy guidelines for documenting functions
\end{enumerate}

More suggestions can be found in PEP8.
%----------------------------%
\end{document}