\documentclass[KSmainSlides.tex]{subfiles} 
\begin{document} 
	%----------------------------------------------------------------------------------------------------%
	
	%	\section{Section 3 : Data Structures}
	
	
	\begin{frame}[fragile]
		\frametitle{Data Structures}	
		\textbf{\textit{pandas}} introduces two new data structures to Python - \textbf{Series} and \textbf{DataFrame}, both of which are built on top of NumPy.
		
		\begin{framed}
			\begin{verbatim}
			import pandas as pd
			import numpy as np
			import matplotlib.pyplot as plt
			pd.set_option('max_columns', 50)
			\end{verbatim}
		\end{framed}
		
	\end{frame}
	%=======================================================================================%
	\begin{frame}[fragile]
		\frametitle{Series}
		
		Series is a one-dimensional labeled array capable of holding any data type (integers, strings, floating point numbers, Python objects, etc.). The axis labels are collectively referred to as the index. The basic method to create a Series is to call:
		
		\begin{framed}
			\begin{verbatim}
			s = Series(data, index=index)
			\end{verbatim}
		\end{framed} 
		Here, data can be many different things:
		
		\begin{itemize}
			\item a Python \texttt{dict}
			\item an \texttt{ndarray}
			\item a scalar value (like 5)
		\end{itemize}
		
	\end{frame}
	%=======================================================================================%
	\begin{frame}[fragile]
		\begin{itemize}
			\item A Series is a one-dimensional object similar to an array, list, or column in a table. 
			\item It will assign a labeled index to each item in the Series. \item By default, each item will receive an index label from 0 to N, where N is the length of the Series minus one.
		\end{itemize}
		
		
		\begin{framed}
			\begin{verbatim}
			# create a Series with an arbitrary list
			s = pd.Series([7, 'Mullingar', 3.1415, -1789710578,
			'Iarmhi Abu'])
			s
			\end{verbatim}
		\end{framed}
		
	\end{frame}
	%=======================================================================================%
	\begin{frame}[fragile]
		\frametitle{Series}
		\textbf{Output from Previous Slide}
		\begin{framed}		
			\begin{verbatim}
			0                7
			1       Mullingar
			2             3.1415
			3      -1789710578
			4    Iarmhi Abu
			dtype: object
			\end{verbatim}
		\end{framed}
	\end{frame}
	%=======================================================================================%
	\begin{frame}[fragile]
		
		Alternatively, you can specify an index to use when creating the Series.
		
		\begin{framed}
			\begin{verbatim}
			s = pd.Series([7, 'Mullingar', 3.1415, -1789710578, 
			'Iarmhi Abu'],
			index=['A', 'Z', 'C', 'Y', 'E'])
			s
			\end{verbatim}
		\end{framed}
		\begin{verbatim}
		A                7
		Z       Mullingar
		C             3.1415
		Y      -1789710578
		E    Iarmhi Abu
		dtype: object
		\end{verbatim}
	\end{frame}
	%=======================================================================================%
	\begin{frame}[fragile]
		\frametitle{Series}		
		The Series constructor can convert a dictonary as well, using the keys of the dictionary as its index.
		
		\begin{framed}
			\begin{verbatim}
			d = {'Chicago': 1000, 'New York': 1300, 'Portland': 900, 'San Francisco': 1100,
			'Austin': 450, 'Boston': None}
			cities = pd.Series(d)
			cities
			Out[4]:
			Austin            450
			Boston            NaN
			Chicago          1000
			New York         1300
			Portland          900
			San Francisco    1100
			dtype: float64
			\end{verbatim}
		\end{framed}
	\end{frame}
	%=======================================================================================%
	\begin{frame}[fragile]
		\frametitle{Series}		
		You can use the index to select specific items from the Series ...
		
		\begin{framed}
			\begin{verbatim}
			cities['Chicago']
			Out[5]:
			1000.0
			\end{verbatim}
		\end{framed}
	\end{frame}
	%=======================================================================================%
	\begin{frame}[fragile]
		\frametitle{Series}		
		\begin{framed}
			\begin{verbatim}
			cities[['Chicago', 'Portland', 'San Francisco']]
			Out[6]:
			Chicago          1000
			Portland          900
			San Francisco    1100
			dtype: float64
			\end{verbatim}
		\end{framed}
	\end{frame}
	%=======================================================================================%
	\begin{frame}[fragile]
		\frametitle{Series}		
		You can use \textbf{\textit{boolean indexing}} for selection.
		
		\begin{framed}
			\begin{verbatim}
			cities[cities < 1000]
			Out[7]:
			Austin      450
			Portland    900
			dtype: float64
			\end{verbatim}
		\end{framed}
	\end{frame}
	%=======================================================================================%
	\begin{frame}[fragile]
		
		That last one might be a little strange, so let's make it more clear - \texttt{cities < 1000} returns a Series of \texttt{True/False} values, which we then pass to our Series cities, returning the corresponding \texttt{True} items.
		
	\end{frame}
	%=======================================================================================%
	\begin{frame}[fragile]
		
		\begin{framed}
			\begin{verbatim}
			less_than_1000 = cities < 1000
			print less_than_1000
			print '\n'
			print cities[less_than_1000]
			Austin            True
			Boston           False
			Chicago          False
			New York         False
			Portland          True
			San Francisco    False
			dtype: bool
			
			
			Austin      450
			Portland    900
			dtype: float64
			
			\end{verbatim}
		\end{framed}
	\end{frame}
	%=======================================================================================%
	\begin{frame}[fragile]
		
		You can also change the values in a Series on the fly.
		
		\begin{framed}
			\begin{verbatim}
			# changing based on the index
			
			print 'Old value:', cities['Chicago']
			
			cities['Chicago'] = 1400
			print 'New value:', cities['Chicago']
			
			Old value: 1000.0
			New value: 1400.0
			\end{verbatim}
		\end{framed}
	\end{frame}
	%=======================================================================================%
	\begin{frame}[fragile]	
		Changing values using boolean logic
		\begin{framed}
			\begin{verbatim}
			
			print cities[cities < 1000]
			print '\n'
			cities[cities < 1000] = 750
			
			print cities[cities < 1000]
			Austin      450
			Portland    900
			dtype: float64
			
			
			Austin      750
			Portland    750
			dtype: float64
			\end{verbatim}
		\end{framed}
	\end{frame}
	%=======================================================================================%
	\begin{frame}[fragile]
		\frametitle{Working with Series}
		What if you aren't sure whether an item is in the Series? You can check using idiomatic Python.
		
		\begin{framed}
			\begin{verbatim}
			print 'Seattle' in cities
			print 'San Francisco' in cities
			False
			True
			\end{verbatim}
		\end{framed}
	\end{frame}
	%=======================================================================================%
	\begin{frame}[fragile]
		Mathematical operations can be done using scalars and functions.
		
		\begin{framed}
			\begin{verbatim}
			# divide city values by 3
			cities / 3
			Out[12]:
			Austin           250.000000
			Boston                  NaN
			Chicago          466.666667
			New York         433.333333
			Portland         250.000000
			San Francisco    366.666667
			dtype: float64
			\end{verbatim}
		\end{framed}
	\end{frame}
	%=======================================================================================%
	\begin{frame}[fragile]
		\begin{framed}
			\begin{verbatim}
			# square city values
			np.square(cities)
			Out[13]:
			Austin            562500
			Boston               NaN
			Chicago          1960000
			New York         1690000
			Portland          562500
			San Francisco    1210000
			dtype: float64
			\end{verbatim}
		\end{framed}
	\end{frame}
	%=======================================================================================%
	\begin{frame}[fragile]
		
		You can add two Series together, which returns a union of the two Series with the addition occurring on the shared index values. Values on either Series that did not have a shared index will produce a NULL/NaN (not a number).
		
		\begin{framed}
			\begin{verbatim}
			print cities[['Chicago', 'New York', 'Portland']]
			print'\n'
			print cities[['Austin', 'New York']]
			print'\n'
			print cities[['Chicago', 'New York', 'Portland']] + cities[['Austin', 'New York']]
			\end{verbatim}
		\end{framed}
	\end{frame}
	
	%=======================================================================================%
	\begin{frame}[fragile]
		\begin{verbatim}
		Chicago     1400
		New York    1300
		Portland     750
		dtype: float64
		
		
		Austin       750
		New York    1300
		dtype: float64
		
		
		Austin       NaN
		Chicago      NaN
		New York    2600
		Portland     NaN
		dtype: float64
		\end{verbatim}
	\end{frame}
	%=======================================================================================%
	\begin{frame}[fragile]
		\frametitle{Working with Series}
		\textbf{NULL Checking}
		\begin{itemize}
			
			\item Notice that because Austin, Chicago, and Portland were not found in both Series, they were returned with NULL/NaN values.
			
			\item NULL checking can be performed with \texttt{isnull()} and \texttt{notnull()}.
		\end{itemize}		
	\end{frame}
	%=======================================================================================%
	\begin{frame}[fragile]
		Return a boolean series indicating which values aren't NULL
		
		\begin{framed}
			\begin{verbatim}
			cities.notnull()
			
			Austin            True
			Boston           False
			Chicago           True
			New York          True
			Portland          True
			San Francisco     True
			dtype: bool
			\end{verbatim}
		\end{framed}
	\end{frame}
	%=======================================================================================%
	%=======================================================================================%
	\begin{frame}[fragile]
		Using boolean logic to grab the NULL cities
		\begin{framed}
			\begin{verbatim}
			print cities.isnull()
			print '\n'
			print cities[cities.isnull()]
			Austin           False
			Boston            True
			Chicago          False
			New York         False
			Portland         False
			San Francisco    False
			dtype: bool
			
			Boston   NaN
			dtype: float64
			\end{verbatim}
		\end{framed}
	\end{frame}
	%=======================================================================================%
\end{document}

%%---------------------------------------%
%\newpage
%\frametitle{DataFrame}
%
%% pandas - chapter 5 - DataFrame
%
%A DataFrame is a tablular data structure comprised of rows and columns, akin to a spreadsheet, database table, or R's data.frame object. You can also think of a DataFrame as a group of Series objects that share an index (the column names).
%
%%For the rest of the tutorial, we'll be primarily working with DataFrames.
%
%%---------------------------------------%
%\newpage
%\frametitle{Panel}
%
%
% 
%\texttt{Panel} is a somewhat less-used, but still important container for 3-dimensional data. 
%The term panel data is derived from econometrics and is partially responsible for the name pandas: pan(el)-da(ta)-s. 
%The names for the 3 axes are intended to give some semantic meaning to describing operations involving panel data and, 
%in particular, econometric analysis of panel data. However, for the strict purposes of slicing and dicing a 
%collection of DataFrame objects, you may find the axis names slightly arbitrary:
% 
%\begin{itemize}
%\item items: axis 0, each item corresponds to a DataFrame contained inside
%\item major\_axis: axis 1, it is the index (rows) of each of the DataFrames
%\item minor\_axis: axis 2, it is the columns of each of the DataFrames
%\end{itemize}
%
%\newpage
%
%\end{document}