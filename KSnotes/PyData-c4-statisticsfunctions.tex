%	19.6 Select Statistical Tests	

\documentclass[KSmain.tex]{subfiles} 
\begin{document} 
	\large
	
%========================================================== %
% 19.3
\subsection{Statistics Functions}
\subsubsection*{\texttt{mean}}
\texttt{mean} computes the average of an array. An optional second argument provides the axis to use (default is
to use entire array). \texttt{mean} can be used either as a function or as a method on an array.
\begin{framed}
	\begin{verbatim}
	>>> x = arange(10.0)
	>>> x.mean()
	4.5
	>>> mean(x)
	4.5
	>>> x= reshape(arange(20.0),(4,5))
	>>> mean(x,0)
	231
	array([ 7.5, 8.5, 9.5, 10.5, 11.5])
	>>> x.mean(1)
	array([ 2., 7., 12., 17.])
	\end{verbatim}
\end{framed}
\subsubsection*{\texttt{median}}
\texttt{median} computed the median value in an array. An optional second argument provides the axis to use
(default is to use entire array).
\begin{framed}
	\begin{verbatim}
	>>> x= randn(4,5)
	>>> x
	array([[0.74448693,
	0.63673031,
	0.40608815,
	0.40529852, 0.93803737],
	[ 0.77746525, 0.33487689, 0.78147524, 0.5050722
	, 0.58048329],
	[0.51451403,
	0.79600763,
	0.92590814, 0.53996231,
	0.24834136],
	[0.83610656,
	0.29678017, 0.66112691,
	0.10792584, 1.23180865]])
	>>> median(x)
	0.45558017286810903
	>>> median(x, 0)
	array([0.62950048,
	0.16997507,
	0.18769355, 0.19857318,
	0.59318936])
	\end{verbatim}
\end{framed}
Note that when an array or axis dimension contains an even number of elements (n), median returns the
average of the 2 inner elements.
%======================%
\subsubsection*{\texttt{std}}
\texttt{std} computes the standard deviation of an array. An optional second argument provides the axis to use
(default is to use entire array). std can be used either as a function or as a method on an array.
\subsubsection*{\texttt{var}}
\texttt{var} computes the variance of an array. An optional second argument provides the axis to


\subsubsection*{\texttt{corrcoef}}
\texttt{corrcoef(x)} computes the correlation between the rows of a 2-dimensional array x . \texttt{corrcoef(x, y)} computes
the correlation between two 1- dimensional vectors. An optional keyword argument rowvar can be
used to compute the correlation between the columns of the input – this is corrcoef(x, rowvar=False)
and \texttt{corrcoef(x.T)} are identical

\begin{framed}
	\begin{verbatim}
	>>> x= randn(3,4)
	>>> corrcoef(x)
	array([[ 1. , 0.36780596, 0.08159501],
	[ 0.36780596, 1. , 0.66841624],
	[ 0.08159501, 0.66841624, 1. ]])
	>>> corrcoef(x[0],x[1])
	array([[ 1. , 0.36780596],
	[ 0.36780596, 1. ]])
	>>> corrcoef(x, rowvar=False)
	array([[ 1. , 0.98221501,
	0.19209871,
	0.81622298],
	[0.98221501,
	1. , 0.37294497, 0.91018215],
	[0.19209871,
	0.37294497, 1. , 0.72377239],
	[0.81622298,
	0.91018215, 0.72377239, 1. ]])
	>>> corrcoef(x.T)
	array([[ 1. , 0.98221501,
	0.19209871,
	0.81622298],
	[0.98221501,
	1. , 0.37294497, 0.91018215],
	[0.19209871,
	0.37294497, 1. , 0.72377239],
	[0.81622298,
	0.91018215, 0.72377239, 1. ]])
	\end{verbatim}
\end{framed}
\subsubsection*{\texttt{cov}}
\texttt{cov(x)} computes the covariance of an array x . \texttt{cov(x,y)} computes the covariance between two 1-dimensional
vectors. An optional keyword argument rowvar can be used to compute the covariance between the
columns of the input – this is \texttt{cov(x, rowvar=False)} and \texttt{cov(x.T)} are identical.
\subsubsection*{\texttt{histogram}}
\texttt{histogram} can be used to compute the histogram (empirical frequency, using k bins) of a set of data. An
optional second argument provides the number of bins. If omitted, \texttt{k =10} bins are used. histogram returns
two outputs, the first with a k-element vector containing the number of observations in each bin, and the
second with the k + 1 endpoints of the k bins.
\begin{framed}
	\begin{verbatim}
	>>> x = randn(1000)
	>>> count, binends = histogram(x)
	>>> count
	array([ 7, 27, 68, 158, 237, 218, 163, 79, 36, 7])
	>>> binends
	array([3.06828057,
	2.46725067,
	1.86622077,
	1.26519086,
	0.66416096,
	0.06313105,
	0.53789885, 1.13892875, 1.73995866, 2.34098856,
	2.94201846])
	>>> count, binends = histogram(x, 25)
	\end{verbatim}
\end{framed}
\subsubsection*{\texttt{histogram2d}}
\texttt{histogram2d(x,y)} computes a 2-dimensional histogram for 1-dimensional vectors. An optional keyword
argument bins provides the number of bins to use. bins can contain either a single scalar integer or a
2-element list or array containing the number of bins to use in each dimension.

\subsection{More Statistical Functions}
\subsubsection*{\texttt{Full Specification}}

For the sake of brevity and space, the functions names only are presented below.
It may be necessary to use their full specifications, depending on how you have specified the relevant \texttt{import} functions.
Examples of full specifications are:
\begin{framed}
\begin{verbatim}
np.random.randint()
np.stats.linregress()
\end{verbatim}
\end{framed}

\subsubsection*{\texttt{mode}}
mode computes the mode of an array. An optional second argument provides the axis to use (default is to
use entire array). Returns two outputs: the first contains the values of the mode, the second contains the
number of occurrences.
\begin{framed}
	\begin{verbatim}
	>>> x=randint(1,11,1000)
	>>> stats.mode(x)
	(array([ 4.]), array([ 112.]))
	\end{verbatim}
\end{framed}
\subsubsection*{\texttt{moment}}
moment computed the rth central moment for an array. An optional second argument provides the axis to
use (default is to use entire array).
\begin{framed}
	\begin{verbatim}
	>>> x = randn(1000)
	>>> moment = stats.moment
	>>> moment(x,2) moment(
	x,1)**2
	0.94668836546169166
	>>> var(x)
	0.94668836546169166
	>>> x = randn(1000,2)
	>>> moment(x,2,0) # axis 0
	array([ 0.97029259, 1.03384203])
	\end{verbatim}
\end{framed}
\subsubsection*{\texttt{skew}}
skew computes the skewness of an array. An optional second argument provides the axis to use (default is
to use entire array).
\begin{framed}
	\begin{verbatim}
	>>> x = randn(1000)
	>>> skew = stats.skew
	>>> skew(x)
	0.027187705042705772
	>>> x = randn(1000,2)
	>>> skew(x,0)
	array([ 0.05790773, 0.00482564])
	\end{verbatim}
\end{framed}
%------------------------------------------------------------%
\subsubsection*{\texttt{kurtosis}}
kurtosis computes the excess kurtosis (actual kurtosis minus 3) of an array. An optional second argument
provides the axis to use (default is to use entire array). Setting the keyword argument fisher=False will
compute the actual kurtosis.
\begin{framed}
	\begin{verbatim}
	>>> x = randn(1000)
	>>> kurtosis = stats.kurtosis
	>>> kurtosis(x)
	0.2112381820194531
	>>> kurtosis(x, fisher=False)
	2.788761817980547
	>>> kurtosis(x, fisher=False) kurtosis(
	x) # Must be 3
	3.0
	>>> x = randn(1000,2)
	>>> kurtosis(x,0)
	array([0.13813704,
	0.08395426])
	\end{verbatim}
\end{framed}
\subsubsection*{\texttt{pearsonr}}
pearsonr computes the Pearson correlation between two 1-dimensional vectors. It also returns the 2-
tailed p-value for the null hypothesis that the correlation is 0.
\begin{framed}
	\begin{verbatim}
	>>> x = randn(10)
	>>> y = x + randn(10)
	>>> pearsonr = stats.pearsonr
	>>> corr, pval = pearsonr(x, y)
	>>> corr
	0.40806165708698366
	>>> pval
	0.24174029858660467
	\end{verbatim}
\end{framed}
%---------------------%
\subsubsection*{\texttt{spearmanr}}
spearmanr computes the Spearmancorrelation (rank correlation). It can be used with a single 2-dimensional
array input, or 2 1-dimensional arrays. Takes an optional keyword argument axis indicating whether to
treat columns (0) or rows (1) as variables. If the input array has more than 2 variables, returns the correlation
matrix. If the input array as 2 variables, returns only the correlation between the variables.
\begin{framed}
	\begin{verbatim}
	>>> x = randn(10,3)
	>>> spearmanr = stats.spearmanr
	>>> rho, pval = spearmanr(x)
	>>> rho
	array([[ 1. , 0.02087009,
	0.05867387],
	[0.02087009,
	1. , 0.21258926],
	[0.05867387,
	0.21258926, 1. ]])
	>>> pval
	array([[ 0. , 0.83671325, 0.56200781],
	[ 0.83671325, 0. , 0.03371181],
	[ 0.56200781, 0.03371181, 0. ]])
	>>> rho, pval = spearmanr(x[:,1],x[:,2])
	>>> corr
	0.020870087008700869
	>>> pval
	0.83671325461864643
	\end{verbatim}
\end{framed}
%---------------------------------------------------------%
\subsubsection*{\texttt{linregress}}
linregress estimates a linear regression between 2 1-dimensional arrays. It takes two inputs, the independent
variables (regressors) and the dependent variable (regressand). Models always include a constant.
\begin{framed}
	\begin{verbatim}
	>>> x = randn(10)
	>>> y = x + randn(10)
	>>> linregress = stats.linregress
	>>> slope, intercept, rvalue, pvalue, stderr = linregress(x,y)
	>>> slope
	1.6976690163576993
	>>> rsquare = rvalue**2
	>>> rsquare
	0.59144988449163494
	>>> x.shape = 10,1
	>>> y.shape = 10,1
	>>> z = hstack((x,y))
	>>> linregress(z) # Alternative form, [x y]
	(1.6976690163576993,
	0.79983724584931648,
	0.76905779008578734,
	0.0093169560056056751,
	0.4988520051409559)
	
	\end{verbatim}
\end{framed}
% 19.6 Select Statistical Tests
\subsection{Select Statistical Tests}
% normaltest
% kstest
% twosample KS test
% Shapiro Test

\subsubsection*{\texttt{normaltest}}
\texttt{normaltest} tests for normality in an array of data. An optional second argument provides the axis to use
(default is to use entire array). Returns the test statistic and the p-value of the test. This test is a small
sample modified version of the Jarque-Bera test statistic.
\subsubsection*{\texttt{kstest}}
\texttt{kstest} implements the Kolmogorov-Smirnov test. Requires two inputs, the data to use in the test and the
distribution, which can be a string or a frozen random variable object. If the distribution is provided as
a string, then any required shape parameters are passed in the third argument using a tuple containing
these parameters, in order.
\begin{verbatim}
>>> x = randn(100)
>>> kstest = stats.kstest
>>> stat, pval = kstest(x, ’norm’)
>>> stat
0.11526423481470172
>>> pval
0.12963296757465059
>>> ncdf = stats.norm().cdf # No () on cdf to get the function
>>> kstest(x, ncdf)
(0.11526423481470172, 0.12963296757465059)
>>> x = gamma.rvs(2, size = 100)
>>> kstest(x, ’gamma’, (2,)) # (2,) contains the shape parameter
(0.079237623453142447, 0.54096739528138205)
>>> gcdf = gamma(2).cdf
>>> kstest(x, gcdf)
(0.079237623453142447, 0.54096739528138205)
\end{verbatim}

\subsubsection{\texttt{ks\_2samp}}
\texttt{ks\_2samp} implements a 2-sample version of the Kolmogorov-Smirnov test. It is called \texttt{ks\_2samp(x,y)}
where both inputs are 1-dimensonal arrays, and returns the test statistic and p-value for the null that
the distribution of x is the same as that of y .
\subsubsection{\texttt{shapiro}}
\texttt{shapiro }implements the Shapiro-Wilk test for normality on a 1-dimensional array of data. It returns the
test statistic and p-value for the null of normality.

\newpage
\section{Statsmodels}
\texttt{Statsmodels} is a Python module that allows users to explore data, estimate statistical models, and perform statistical tests. 
An extensive list of descriptive statistics, statistical tests, plotting functions, and result statistics are available for different types of 
data and each estimator. Researchers across fields may find that statsmodels fully meets their needs for statistical computing and data analysis 
in Python. 

Features include:


\begin{itemize}

\item Linear regression models

\item Generalized linear models

\item Discrete choice models

\item Robust linear models

\item Many models and functions for time series analysis

\item Nonparametric estimators

\item A collection of datasets for examples

\item A wide range of statistical tests

\item Input-output tools for producing tables in a number of formats (Text, LaTex, HTML) and for reading Stata files into NumPy and Pandas.

\item Plotting functions

\item Extensive unit tests to ensure correctness of results

\item Many more models and extensions in development

\end{itemize}

\end{document}
