%	19.1 Simulating Random Variables	
%		19.1.1 Core Random Number Generators
%		19.1.2 Random Array Functions
%		19.1.3 Select Random Number Generators
%	19.2 Simulation and Random Number Generation	
%		19.2.1 State
%		19.2.2 Seed
%		19.2.3 Replicating Simulation Data
%	19.3 Statistics Functions	
%	19.4 Continuous Random Variables	
%		19.4.1 Example: gamma
%		19.4.2 Important Distributions
%		19.4.3 Frozen Random Variable Object
%	19.5 Select Statistics Functions	
%	19.6 Select Statistical Tests	

\documentclass[KSmain.tex]{subfiles} 
\begin{document} 

\subsection{Random Number Generation with NumPy}
% 19.1 Simulating Random Variables
% - 19.1.1 Core Random Number Generators
NumPy random number generators are all stored in the module numpy.random. These can be imported with using import numpy as np and then calling np.random.rand, for example, or by importing import numpy.random as rnd andusing rnd.rand.1
\subsubsection{rand, random\_sample}
randandrandom\_sample are uniform randomnumber generators whichare
 identicalexceptthat rand takes a variable number 
of integer inputs – one for each dimension – while random\_sample takes a n-element tuple. 

random\_sample is the preferred NumPy function, and rand is a convenience function primarily for MATLABusers.
\begin{framed}
\begin{verbatim}
>>> x = rand(3,4,5) >>> y = random_sample((3,4,5))
\end{verbatim}
\end{framed}

% 19.1.2 Random Array Functions
\subsubsection{shuffle}
shuffle randomlyreorderstheelementsofanarrayinplace.
\begin{verbatim}
>>> x = arange(10) >>> shuffle(x) >>> x array([4, 6, 3, 7, 9, 0, 2, 1, 8, 5])
\end{verbatim}
\subsubsection{permutation}
permutation returns randomly reordered elements of an array as a copy while not directly changing the input.
\begin{framed}
\begin{verbatim}
>>> x = arange(10) >>> permutation(x) array([2, 5, 3, 0, 6, 1, 9, 8, 4, 7])
>>> x array([0, 1, 2, 3, 4, 5, 6, 7, 8, 9])
\end{verbatim}
\end{framed}
% 19.1.3 Select Random Number Generators
\subsection{Select Random Number Generators}
% NumPyprovidesalargeselectionofrandomnumbergeneratorsforspecificdistribution. 

All takebetween 0 and 2 required inputs which are parameters of the distribution, plus a tuple containing the size of the output. Allrandomnumbergeneratorsareinthemodule 
numpy.random.

% \subsection{Bernoulli}
% ThereisnoBernoulligenerator. Instead usebinomial(1,p)to generateasingle draworbinomial(1,p,(10,10)) togenerate anarray where % p istheprobabilityofsuccess.




%=========================================================%
%19.3 - Statistical Functions

% - Median
% - Standard Deviation
% - Variance
% - Correlation
% - Covariance
% - Histograms
% - Histogram Plots

\subsubsection{median}
median computed the median value in an array. An optional second argument provides the axis to use
(default is to use entire array).
\begin{verbatim}
>>> x= randn(4,5)
>>> x
array([[0.74448693,
0.63673031,
0.40608815,
0.40529852, 0.93803737],
[ 0.77746525, 0.33487689, 0.78147524, 0.5050722
, 0.58048329],
[0.51451403,
0.79600763,
0.92590814, 0.53996231,
0.24834136],
[0.83610656,
0.29678017, 0.66112691,
0.10792584, 1.23180865]])
>>> median(x)
0.45558017286810903
>>> median(x, 0)
array([0.62950048,
0.16997507,
0.18769355, 0.19857318,
0.59318936])
\end{verbatim}
Note that when an array or axis dimension contains an even number of elements (n), median returns the
average of the 2 inner elements.
%======================%
\subsubsection{std}
std computes the standard deviation of an array. An optional second argument provides the axis to use
(default is to use entire array). std can be used either as a function or as a method on an array.
\subsubsection{var}
var computes the variance of an array. An optional second argument provides the axis to

....

\subsubsection{corrcoef}
corrcoef(x) computes the correlation between the rows of a 2-dimensional array x . corrcoef(x, y) computes
the correlation between two 1- dimensional vectors. An optional keyword argument rowvar can be
used to compute the correlation between the columns of the input – this is corrcoef(x, rowvar=False)
and corrcoef(x.T) are identical



\subsubsection{cov}
cov(x) computes the covariance of an array x . cov(x,y) computes the covariance between two 1-dimensional
vectors. An optional keyword argument rowvar can be used to compute the covariance between the
columns of the input – this is cov(x, rowvar=False) and cov(x.T) are identical.
histogram
histogram can be used to compute the histogram (empirical frequency, using k bins) of a set of data. An
optional second argument provides the number of bins. If omitted, k =10 bins are used. histogram returns
two outputs, the first with a k-element vector containing the number of observations in each bin, and the
second with the k + 1 endpoints of the k bins.
%=========================================================%
\newpage
% 19.6 Select Statistical Tests
% normaltest
% kstest
% twosample KS test
% Shapiro Test

\subsection{normaltest}
normaltest tests for normality in an array of data. An optional second argument provides the axis to use
(default is to use entire array). Returns the test statistic and the p-value of the test. This test is a small
sample modified version of the Jarque-Bera test statistic.
\subsection{kstest}
kstest implements the Kolmogorov-Smirnov test. Requires two inputs, the data to use in the test and the
distribution, which can be a string or a frozen random variable object. If the distribution is provided as
a string, then any required shape parameters are passed in the third argument using a tuple containing
these parameters, in order.
\begin{verbatim}
>>> x = randn(100)
>>> kstest = stats.kstest
>>> stat, pval = kstest(x, ’norm’)
>>> stat
0.11526423481470172
>>> pval
0.12963296757465059
>>> ncdf = stats.norm().cdf # No () on cdf to get the function
>>> kstest(x, ncdf)
(0.11526423481470172, 0.12963296757465059)
>>> x = gamma.rvs(2, size = 100)
>>> kstest(x, ’gamma’, (2,)) # (2,) contains the shape parameter
(0.079237623453142447, 0.54096739528138205)
>>> gcdf = gamma(2).cdf
>>> kstest(x, gcdf)
(0.079237623453142447, 0.54096739528138205)
\end{verbatim}

\subsubsection{ks\_2samp}
ks\_2samp implements a 2-sample version of the Kolmogorov-Smirnov test. It is called ks\_2samp(x,y)
where both inputs are 1-dimensonal arrays, and returns the test statistic and p-value for the null that
the distribution of x is the same as that of y .
shapiro
shapiro implements the Shapiro-Wilk test for normality on a 1-dimensional array of data. It returns the
test statistic and p-value for the null of normality.

\newpage
\section{Statsmodels}
Statsmodels is a Python module that allows users to explore data, estimate statistical models, and perform statistical tests. 
An extensive list of descriptive statistics, statistical tests, plotting functions, and result statistics are available for different types of 
data and each estimator. Researchers across fields may find that statsmodels fully meets their needs for statistical computing and data analysis 
in Python. Features include:


\begin{itemize}

\item Linear regression models

\item Generalized linear models

\item Discrete choice models

\item Robust linear models

\item Many models and functions for time series analysis

\item Nonparametric estimators

\item A collection of datasets for examples

\item A wide range of statistical tests

\item Input-output tools for producing tables in a number of formats (Text, LaTex, HTML) and for reading Stata files into NumPy and Pandas.

\item Plotting functions

\item Extensive unit tests to ensure correctness of results

\item Many more models and extensions in development

\end{itemize}

\end{document}
