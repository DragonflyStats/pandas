%	4.1 Array	
%		4.1.1 Array dtypes
%	4.2 Matrix	
%		4.3 1-dimensional Arrays
%	4.4 2-dimensional Arrays	
%	4.5 Multidimensional Arrays	
%	4.6 Concatenation	
%	4.7 Accessing Elements of an Array	
%		4.7.1 Scalar Selection
%		4.7.2 Array Slicing
%		4.7.3 Mixed Selection using Scalar and Slice Selectors
%		4.7.4 Assignment using Slicing
%		4.7.5 Linear Slicing using flat
%	4.8 Slicing and Memory Management	
%	4.9 import and Modules	
%	4.10 Calling Functions	
%--------------------------------------------------------------------------------------------%


\documentclass[KSmain.tex]{subfiles} 
\begin{document} 

\section{Arrays}
Arrays are the base data type in NumPy, are are arrays in some ways similar to lists since they both contain
collections of elements. The focus of this section is on homogeneous arrays containing numeric data
– that is, an array where all elements have the same numeric type (heterogeneous arrays are covered in
Chapters 16 and 17). 

Additionally, arrays, unlike lists, are always rectangular so that all rows have the same
number of elements.

Arrays are initialized from lists (or tuples) using array. Two-dimensional arrays are initialized using
lists of lists (or tuples of tuples, or lists of tuples, etc.), and higher dimensional arrays can be initialized by
further nesting lists or tuples.

%=======================================%
\subsection{Matrix}
Matrices are essentially a subset of arrays, and behave in a virtually identical manner. The two important
differences are:

\begin{itemize}
\item Matrices always have 2 dimensions
\item Matrices follow the rules of linear algebra for $\ast$
\end{itemize}

\newpage
%=======================================%

\subsection{Accessing Elements of an Array}
Four methods are available for accessing elements contained within an array: scalar selection, slicing,
numerical indexing and logical (or Boolean) indexing. Scalar selection and slicing are the simplest and so
are presented first. Numerical indexing and logical indexing both depends on specialized functions and
so these methods are discussed in Chapter 12



%-------------------------------------------------------------------------%

\newpage

4.4 2-dimensional Arrays
Matrices and 2-dimensional arrays are rows of columns, and so
x =
2
64
1 2 3
4 5 6
7 8 9
3
75
,
is input by enter the matrix one row at a time, each in a list, and then encapsulate the row lists in another
list.
>>> x = array([[1.0,2.0,3.0],[4.0,5.0,6.0],[7.0,8.0,9.0]])
>>> x
array([[ 1., 2., 3.],
[ 4., 5., 6.],
[ 7., 8., 9.]])
%---------------------------------------------%
% KS - 4.5 Multidimensional Arrays
\newpage
\subsection{Multidimensional Arrays}
Higher dimensional arrays are useful when tracking matrix valued processes through time, such as a timevarying
covariance matrices. Multidimensional (N -dimensional) arrays are available forN up to about 30,
depending on the size of each matrix dimension. Manually initializing higher dimension arrays is tedious
and error prone, and so it is better to use functions such as zeros((2, 2, 2)) or empty((2, 2, 2)).
%---------------------------------------------%
\newpage
\subsection{Concatenation}
% KS -  4.6 Concatenation
Concatenation is the process by which one vector or matrix is appended to another. Arrays and matrices
can be concatenation horizontally or vertically
%---------------------------------------------%
4.7 Accessing Elements of an Array
Four methods are available for accessing elements contained within an array: scalar selection, slicing,
numerical indexing and logical (or Boolean) indexing. Scalar selection and slicing are the simplest and so
are presented first. Numerical indexing and logical indexing both depends on specialized functions and
so these methods are discussed in Chapter 12.
%-------------------------------------------------------------------------%
% KS page 59
% KS - 4.9 import and Modules
\newpage
\subsection{The \texttt{import} function}
Python, by default, only has access to a small number of built-in types and functions. The vast majority of
functions are located in modules, and before a function can be accessed, the module which contains the
function must be imported. 

For example, when using ipython --pylab (or any variants), a large number
of modules are automatically imported, including NumPy and matplotlib. This is style of importing useful
for learning and interactive use, but care is needed to make sure that the correct module is imported when
designing more complex programs.

import can be used in a variety of ways. The simplest is to use from module import * which imports
all functions in module. This method of using import can dangerous since if you use it more than once,
it is possible for functions to be hidden by later imports. A better method is to just import the required
functions. This still places functions at the top level of the namespace, but can be used to avoid conflicts.
\begin{verbatim}
from pylab import log2 # Will import log2 only
from scipy import log10 # Will not import the log2 from SciPy
\end{verbatim}

%KS page 59
The only difference between these two is that import scipy is implicitly calling import scipy as scipy.
When this formof import is used, functions are used with the “as” name. For example, the load provided
byNumPy is accessed using sp.log2, while the pylab load is pl.log2 – and both can be used where appropriate.
While this method is the most general, it does require slightly more typing.
%--------------------------------------------------------------------------------------------%
\end{document}
