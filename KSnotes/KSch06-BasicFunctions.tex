\section{Basic Functions and Numerical Indexing}
% 6.3 Mathematics
% 6.2 Rounding and Precision
% 6.1 Moved to last
%=================================%
\subsection{Mathematics}
\subsubsection*{sum, cumsum}
sum sums elements in an array. By default, it will sum all elements in the array, and so the second argument
is normally used to provide the axis to use – 0 tosumdowncolumns, 1 tosumacross rows. cumsum produces
the cumulativesumof the values in the array, and is also usually used with the second argument to indicate
the axis to use.
%----------------%
\begin{framed}
\begin{verbatim}
>>> x = randn(3,4)
>>> x
array([[0.08542071,
2.05598312,
2.1114733 , 0.7986635 ],
[0.17576066,
0.83327885, 0.64064119,
0.25631728],
[0.38226593,
1.09519101,
0.29416551, 0.03059909]])
>>> sum(x) # all elements
0.62339964288008698
>>> sum(x, 0) # Down rows, 4 elements
array([0.6434473
, 2.31789529,
1.76499762, 0.57294532])
>>> sum(x, 1) # Across columns, 3 elements
array([ 0.76873297, 0.23944028,
1.15269233])
>>> cumsum(x,0) # Down rows
array([[0.08542071,
2.05598312,
2.1114733 , 0.7986635 ],
[0.26118137,
1.22270427,
1.47083211, 0.54234622],
[0.6434473
, 2.31789529,
1.76499762, 0.57294532]])
\end{verbatim}
\end{framed}
%----------------------%
sum and cumsum can both be used as function or as methods. When used as methods, the first input is the
axis so that sum(x,0) is the same as x.sum(0).
\subsubsection*{prod, cumprod}
prod and cumprod behave similarly to sum and cumsum except that the product and cumulative product are
returned. prod and cumprod can be called as function or methods.

\subsubsection*{diff}
diff computes the finite difference of a vector (also array) and returnsn-1 an element vectorwhenused on
an n element vector. diff operates on the last axis by default, and so diff(x) operates across columns and
returns x[:,1:size(x,1)]x[:,:
size(x,1)1]
for a 2-dimensional array. diff takes an optional keyword
75
argument axis so that diff(x, axis=0) will operate across rows. diff can also be used to produce higher
order differences (e.g. double difference).
%----------------%
\begin{framed}
\begin{verbatim}>>> x= randn(3,4)
>>> x
array([[0.08542071,
2.05598312,
2.1114733 , 0.7986635 ],
[0.17576066,
0.83327885, 0.64064119,
0.25631728],
[0.38226593,
1.09519101,
0.29416551, 0.03059909]])
>>> diff(x) # Same as diff(x,1)
0.62339964288008698
>>> diff(x, axis=0)
array([[0.09033996,
2.88926197, 2.75211449,
1.05498078],
[0.20650526,
1.92846986,
0.9348067 , 0.28691637]])
>>> diff(x, 2, axis=0) # Double difference, columnbycolumn
array([[0.11616531,
4.81773183,
3.68692119, 1.34189715]])
\end{verbatim}
\end{framed}
%----------------------%
\subsubsection*{exp}
exp returns the element-by-element exponential ($e^x$) for an array.
\subsubsection*{log}
log returns the element-by-element natural logarithm (ln(x )) for an array.
\subsubsection*{log10}
log10 returns the element-by-element base-10 logarithm (log10 (x )) for an array.

%====================================%
\subsection{6.2 Rounding}
\subsubsection*{around, round}
around rounds to the nearest integer, or to a particular decimal place when called with two arguments.
\begin{framed}
\begin{verbatim}
%----------------%
\begin{framed}
\begin{verbatim}
>>> x = randn(3)
array([ 0.60675173, 0.3361189
, 0.56688485])
>>> around(x)
array([ 1., 0., 1.])
>>> around(x, 2)
array([ 0.61, 0.34,
0.57])
\end{verbatim}
\end{framed}
%----------------------%
around can also be used as a method on an ndarray – except that the method is named round. For example,
x.round(2) is identical to around(x, 2). The change of names is needed to avoid conflicting with the
Python built-in function round.
\subsubsection*{floor}
floor rounds to the next smallest integer.
>>> x = randn(3)
array([ 0.60675173, 0.3361189
, 0.56688485])
>>> floor(x)
array([ 0., 1.,
1.])
\end{verbatim}
\end{framed}
%----------------------%
\subsubsection*{ceil}
ceil rounds to the next largest integer.
%----------------%
\begin{framed}
\begin{verbatim}
>>> x = randn(3)
array([ 0.60675173, 0.3361189
, 0.56688485])
>>> ceil(x)
array([ 1., 0.,
0.])
\end{verbatim}
\end{framed}
%----------------------%
Note that the values returned are still floating points and so 0.
is the same as 0..

%========================================%
% Section 6.1


\subsection{Generating Arrays and Matrices}
\subsubection{linspace}
linspace(l,u,n) generates a set of n points uniformly spaced between l, a lower bound (inclusive) and u,
an upper bound (inclusive).
\begin{framed}
\begin{verbatim}
>>> x = linspace(0, 10, 11)
>>> x
array([ 0., 1., 2., 3., 4., 5., 6., 7., 8., 9., 10.])
\end{verbatim}
\end{framed}
%----------------------------------%
\subsubection{logspace}
logspace(l,u,n) produces a set of logarithmically spaced points between 10l and 10u . It is identical to
10**linspace(l,u,n).
%----------------------------------%
\subsubection{arange
arange(l,u,s) produces a set of points spaced by s between l, a lower bound (inclusive) and u, an upper
bound (exclusive). arange can be used with a single parameter, so that arange(n) is equivalent to
arange(0,n,1). Note that arange will return integer data type if all inputs are integer.
\begin{framed}
\begin{verbatim}
>>> x = arange(11)
array([ 0, 1, 2, 3, 4, 5, 6, 7, 8, 9, 10])
>>> x = arange(11.0)
array([ 0., 1., 2., 3., 4., 5., 6., 7., 8., 9., 10.])
>>> x = arange(4, 10, 1.25)
array([ 4. , 5.25, 6.5 , 7.75, 9. ])
\end{verbatim}
\end{framed}
%----------------------------------%
\subsubection{meshgrid}
meshgrid broadcasts two vectors to produce two 2-dimensional arrays, and is a useful function when plotting
3-dimensional functions.
\begin{framed}
\begin{verbatim}
>>> x = arange(5)
>>> y = arange(3)
>>> X,Y = meshgrid(x,y)
>>> X
array([[0, 1, 2, 3, 4],
[0, 1, 2, 3, 4],
[0, 1, 2, 3, 4]])
>>> Y
array([[0, 0, 0, 0, 0],
[1, 1, 1, 1, 1],
[2, 2, 2, 2, 2]])
\end{verbatim}
\end{framed}
%-----------------------------------------------%
\subsubsection{r\_}
r\_ is a convenience function which generates 1-dimensional arrays from slice notation. While r\_ is highly
flexible, the most common use it r_[ start : end : stepOrCount ] where start and end are the start and end
points, and stepOrCount can be either a step size, if a real value, or a count, if complex.
\begin{framed}
\begin{verbatim}
>>> r_[0:10:1] # arange equiv
array([0, 1, 2, 3, 4, 5, 6, 7, 8, 9])
>>> r_[0:10:.5] # arange equiv
array([ 0. , 0.5, 1. , 1.5, 2. , 2.5, 3. , 3.5, 4. , 4.5, 5. ,
5.5, 6. , 6.5, 7. , 7.5, 8. , 8.5, 9. , 9.5])
>>> r_[0:10:5j] # linspace equiv, includes end point
array([ 0. , 2.5, 5. , 7.5, 10. ])
\end{verbatim}
\end{framed}
r_ can also be used to concatenate slices using commas to separate slice notation blocks.
\begin{framed}
\begin{verbatim}
>>> r_[0:2, 7:11, 1:4]
array([ 0, 1, 7, 8, 9, 10, 1, 2, 3])
\end{verbatim}
\end{framed}
Note that r\_ is not a function and that is used with [].
%----------------------------------%
\subsubsection{c_}
c\_ is virtually identical to r\_ except that column arrays are generates, which are 2-dimensional (second
dimension has size 1)
\begin{framed}
\begin{verbatim}
>>> c_[0:5:2]
array([[0],
[2],
[4]])
>>> c_[1:5:4j]
array([[ 1. ],
[ 2.33333333],
[ 3.66666667],
[ 5. ]])
\end{verbatim}
\end{framed}
c\_, like r\_, is not a function and is used with [].

%=============================================================================%
\subsubsection{ix_}
ix\_(a,b) constructs an n-dimensional open mesh from n 1-dimensional lists or arrays. The output of
ix\_ is an n-element tuple containing 1-dimensional arrays. The primary use of ix\_ is to simplify selecting
slabs inside a matrix. Slicing can also be used to select elements froman array as long as the slice pattern is
regular. ix\_ is particularly useful for selecting elements froman array using indices which are not regularly
spaced, as in the final example.
\begin{framed}
\begin{verbatim}
>>> x = reshape(arange(25.0),(5,5))
>>> x
array([[ 0., 1., 2., 3., 4.],
[ 5., 6., 7., 8., 9.],
[ 10., 11., 12., 13., 14.],
[ 15., 16., 17., 18., 19.],
[ 20., 21., 22., 23., 24.]])
>>> x[ix_([2,3],[0,1,2])] # Rows 2 & 3, cols 0, 1 and 2
array([[ 10., 11., 12.],
[ 15., 16., 17.]])
>>> x[2:4,:3] # Same, standard slice
array([[ 10., 11., 12.],
[ 15., 16., 17.]])
>>> x[ix_([0,3],[0,1,4])] # No slice equiv
\end{verbatim}
\end{framed}
%=====================%
\subsubsection{mgrid}
mgrid is very similar to meshgrid but behaves like r\_ and c\_ in that it takes slices as input, and uses a
real valued variable to denote step size and complex to denote number of values. The output is an n + 1
dimensional vector where the first index of the output indexes the meshes.
\begin{framed}
\begin{verbatim}
>>> mgrid[0:3,0:2:.5]
array([[[ 0. , 0. , 0. , 0. ],
[ 1. , 1. , 1. , 1. ],
[ 2. , 2. , 2. , 2. ]],
[[ 0. , 0.5, 1. , 1.5],
[ 0. , 0.5, 1. , 1.5],
[ 0. , 0.5, 1. , 1.5]]])
>>> mgrid[0:3:3j,0:2:5j]
array([[[ 0. , 0. , 0. , 0. , 0. ],
[ 1.5, 1.5, 1.5, 1.5, 1.5],
[ 3. , 3. , 3. , 3. , 3. ]],
[[ 0. , 0.5, 1. , 1.5, 2. ],
[ 0. , 0.5, 1. , 1.5, 2. ],
[ 0. , 0.5, 1. , 1.5, 2. ]]])
\end{verbatim}
\end{framed}

\subsubsection{ogrid}
ogrid is identical to mgrid except that the arrays returned are always 1-dimensional. ogrid output is generally
more appropriate for looping code, while mgrid is usually more appropriate for vectorized code.
When the size of the arrays is large, then ogrid uses much less memory.
\begin{framed}
\begin{verbatim}
>>> ogrid[0:3,0:2:.5]
[array([[ 0.],
[ 1.],
[ 2.]]), array([[ 0. , 0.5, 1. , 1.5]])]
>>> ogrid[0:3:3j,0:2:5j]
[array([[ 0. ],
[ 1.5],
[ 3. ]]),
array([[ 0. , 0.5, 1. , 1.5, 2. ]])]
\end{verbatim}
\end{framed}
\end{document}
