There are two kinds of standard deviations (SD): the population SD and the sample SD.
%--------------------------------%
The population SD



is used when the values represent the entire universe of values that you are studying.

%--------------------------------%
The sample SD



is used when the values are a mere sample from that universe.

np.std calculates the population SD by default, while Pandas' Series.std calculates the sample SD by default.
%--------------------------------%

In [42]: np.std([4,5])
Out[42]: 0.5

In [43]: np.std([4,5], ddof=0)
Out[43]: 0.5

In [44]: np.std([4,5], ddof=1)
Out[44]: 0.70710678118654757

In [45]: x = pd.Series([4,5])

In [46]: x.std()
Out[46]: 0.70710678118654757

In [47]: x.std(ddof=0)
Out[47]: 0.5
ddof stands for "degrees of freedom", and controls the number subtracted from N in the SD formulas.

The formula images above come from this Wikipedia page. There the "uncorrected sample standard deviation" is what I called the population SD, and the "corrected sample standard deviation" is the sample SD.
