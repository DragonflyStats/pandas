Correlation Coefficient

How well does your regression equation truly represent
your set of data?
  One of the ways to determine the answer to this question is to 
exam the  correlation coefficient and the coefficient of determination.


 The correlation coefficient, r, and
 the coefficient of determination, r 2 ,
will appear on the screen that shows the regression equation information
 (be sure the Diagnostics are turned on ---
2nd Catalog (above 0), arrow down to 
DiagnosticOn, press ENTER twice.)
 


In addition to appearing with the regression information, the values r and r 2 can be found under VARS, #5 Statistics → EQ #7 r and #8 r 2 . 
 

 

Correlation Coefficient, r : 

   The quantity r, called the linear correlation coefficient, measures the strength and 
      the direction of a linear relationship between two variables. The linear correlation
       coefficient is sometimes referred to as the Pearson product moment correlation coefficient in
       honor of its developer Karl Pearson.
   The mathematical formula for computing r is:
                             
                                   where n is the number of pairs of data.
           (Aren't you glad you have a graphing calculator that computes this formula?)
   The value of r is such that -1 < r < +1.  The + and – signs are used for positive
      linear correlations and negative linear correlations, respectively.  
   Positive correlation:    If x and y have a strong positive linear correlation, r is close
      to +1.  An r value of exactly +1 indicates a perfect positive fit.   Positive values
      indicate a relationship between x and y variables such that as values for x increases,
      values for  y also increase. 
   Negative correlation:   If x and y have a strong negative linear correlation, r is close
     to -1.  An r value of exactly -1 indicates a perfect negative fit.   Negative values
     indicate a relationship between x and y such that as values for x increase, values
     for y decrease. 
   No correlation:  If there is no linear correlation or a weak linear correlation, r is
     close to 0.  A value near zero means that there is a random, nonlinear relationship
     between the two variables
   Note that r is a dimensionless quantity; that is, it does not depend on the units 
     employed.
   A perfect correlation of ± 1 occurs only when the data points all lie exactly on a
     straight line.  If r = +1, the slope of this line is positive.  If r = -1, the slope of this
     line is negative.  
   A correlation greater than 0.8 is generally described as strong, whereas a correlation
      less than 0.5 is generally described as weak.  These values can vary based upon the
     "type" of data being examined.  A study utilizing scientific data may require a stronger
      correlation than a study using social science data.   


 

Coefficient of Determination, r 2  or  R2 : 

   The coefficient of determination, r 2, is useful because it gives the proportion of 
      the variance (fluctuation) of one variable that is predictable from the other variable.
      It is a measure that allows us to determine how certain one can be in making
      predictions from a certain model/graph.
   The coefficient of determination is the ratio of the explained variation to the total
      variation.
   The coefficient of determination is such that 0 <  r 2 < 1,  and denotes the strength
      of the linear association between x and y.  
   The coefficient of determination represents the percent of the data that is the closest
      to the line of best fit.  For example, if r = 0.922, then r 2 = 0.850, which means that
      85% of the total variation in y can be explained by the linear relationship between x
      and y (as described by the regression equation).  The other 15% of the total variation
      in y remains unexplained.
   The coefficient of determination is a measure of how well the regression line
      represents the data.  If the regression line passes exactly through every point on the
      scatter plot, it would be able to explain all of the variation. The further the line is
      away from the points, the less it is able to explain. 
 
