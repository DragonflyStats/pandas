The quantity r, called the linear correlation coefficient, measures the strength and 
      the direction of a linear relationship between two variables. The linear correlation
       coefficient is sometimes referred to as the Pearson product moment correlation coefficient in
       honor of its developer Karl Pearson.
   The mathematical formula for computing r is:
                             
                                   where n is the number of pairs of data.


%-------------------------------------------------------------%

The value of r is such that -1 < r < +1.  The + and – signs are used for positive
      linear correlations and negative linear correlations, respectively.  
   Positive correlation:    If x and y have a strong positive linear correlation, r is close
      to +1.  An r value of exactly +1 indicates a perfect positive fit.   Positive values
      indicate a relationship between x and y variables such that as values for x increases,
      values for  y also increase. 
   Negative correlation:   If x and y have a strong negative linear correlation, r is close
     to -1.  An r value of exactly -1 indicates a perfect negative fit.   Negative values
     indicate a relationship between x and y such that as values for x increase, values
     for y decrease. 
   No correlation:  If there is no linear correlation or a weak linear correlation, r is
     close to 0.  A value near zero means that there is a random, nonlinear relationship
     between the two variables
   Note that r is a dimensionless quantity; that is, it does not depend on the units 
     employed.
   A perfect correlation of ± 1 occurs only when the data points all lie exactly on a
     straight line.  If r = +1, the slope of this line is positive.  If r = -1, the slope of this
     line is negative.  


%-------------------------------------------------------------%




The formula for ρ is:
 
where, is the covariance, is the standard deviation of ,



%-------------------------------------------------------------%
np.corrcoef

Parameters :


x : array_like
 


A 1-D or 2-D array containing multiple variables and observations. 
Each row of m represents a variable, and each column a single observation of all those variables. Also see rowvar below.
 
y : array_like, optional
 


An additional set of variables and observations. y has the same shape as m.
 

%-------------------------------------------------------------%


\begin{framed}
\begin{verbatim}

>>> np.correlate([1, 2, 3], [0, 1, 0.5])
array([ 3.5])

>>> np.correlate([1, 2, 3], [0, 1, 0.5], "same")
array([ 2. ,  3.5,  3. ])

>>> np.correlate([1, 2, 3], [0, 1, 0.5], "full")
array([ 0.5,  2. ,  3.5,  3. ,  0. ])


\end{verbatim}
\end{framed}
