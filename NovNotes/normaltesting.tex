


"Traditionally, in statistics, you need a p-value of less than 0.05 to reject the null hypothesis." - 


In [12]: import scipy.stats as stats

In [13]: x = stats.norm.rvs(size = 100)

In [14]: stats.normaltest(x)
Out[14]: (1.627533590094232, 0.44318552909231262)
normaltest returns a 2-tuple of the chi-squared statistic, and the associated p-value. Given the null hypothesis that x came from a normal distribution, the p-value represents the probability that a chi-squared statistic that large (or larger) would be seen.

If the p-val is very small, it means it is unlikely that the data came from a normal distribution. For example:
In [15]: y = stats.uniform.rvs(size = 100)

In [16]: stats.normaltest(y)
Out[16]: (31.487039026711866, 1.4543748291516241e-07)
