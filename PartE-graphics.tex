
\documentclass[11pt]{article} % use larger type; default would be 10pt

\usepackage[utf8]{inputenc} 
\usepackage{geometry} % to change the page dimensions
\geometry{a4paper} 
\usepackage{graphicx} 
\usepackage{booktabs} % for much better looking tables
\usepackage{array} % for better arrays (eg matrices) in maths
\usepackage{paralist} % very flexible & customisable lists (eg. enumerate/itemize, etc.)
\usepackage{verbatim} % adds environment for commenting out blocks of text & for better verbatim
\usepackage{subfig} 
\usepackage{framed}
\usepackage{subfiles}
\usepackage{fancyhdr} % This should be set AFTER setting up the page geometry
\pagestyle{fancy} % options: empty , plain , fancy
\renewcommand{\headrulewidth}{0pt} % customise the layout...
\lhead{}\chead{Data Analysis with Python}\rhead{}
\lfoot{}\cfoot{\thepage}\rfoot{}
%--------------------------------------------------------------------------------------------%
\usepackage{sectsty}
\allsectionsfont{\sffamily\mdseries\upshape} 
\usepackage[nottoc,notlof,notlot]{tocbibind} % Put the bibliography in the ToC
\usepackage[titles,subfigure]{tocloft} % Alter the style of the Table of Contents
\renewcommand{\cftsecfont}{\rmfamily\mdseries\upshape}
\renewcommand{\cftsecpagefont}{\rmfamily\mdseries\upshape} % No bold!
%--------------------------------------------------------------------------------------------%

\title{Brief Article}
\author{The Author}
%--------------------------------------------------------------------------------------------%

\begin{document}
%\maketitle
%--------------------------------------------------------------------------------------------%
\section{Plotting with Python}

%--------------------------------------------------------------------------------------------%
\subsection{\texttt{matlibplot}}
\begin{itemize}
\item Matplotlib is a complete plotting library capable of high-quality graphics. Matplotlib contains both high level functions which produce specific types of figures, for example a simple line plot or a bar chart, as
well as a low level API for creating highly customized charts. 

\item This chapter covers the basics of producing
plots and only scratches the surface of the capabilities of matplotlib. 
\item Further information is available on
the matplotlib website or in books dedicated to producing print quality graphics using matplotlib.
\end{itemize}
\subsection{\texttt{seaborn}}

seaborn is a Python package which provides a number of advanced data visualized plots. It also provides a
general improvement in the default appearance of matplotlib-produced plots, and so I recommend using
it by default.

\begin{framed}
\begin{verbatim}
 import seaborn as sns

\end{verbatim}
\end{framed}
All figure in this chapter were produced with seaborn loaded, using the default options. The dark grid
background can be swapped to a light grid or no grid using sns.set(stype=’whitegrid’) (light grid) or
sns.set(stype=’nogrid’) (no grid, most similar to matplotlib).

\subsection{Histograms}

Histograms can be produced using hist. A basic histogram produced using the code below is presented
in Figure 15.5, panel (a). This example sets the number of bins used in producing the histogram using the
keyword argument bins.

\subsection{Adding a Title and Legend}
Titles are added with title and legends are added with legend. legend requires that lines have labels,
which is why 3 calls are made to plot – each series has its own label. Executing the next code block produces
a the image in figure 15.8, panel (a).

\begin{framed}
\begin{verbatim}
>>> x = cumsum(randn(100,3), axis = 0)
>>> plot(x[:,0],’b’,
label = ’Series 1’)
>>> hold(True)
>>> plot(x[:,1],’g.’,
label = ’Series 2’)
>>> plot(x[:,2],’r:’,label = ’Series 3’)
>>> legend()
>>> title(’Basic Legend’)
\end{verbatim}
\end{framed}


legend takes keyword arguments which can be used to change its location (loc and an integer, see the
docstring), remove the frame (frameon) and add a title to the legend box (title). The output of a simple
example using these options is presented in panel (b).



\begin{framed}
\begin{verbatim}
>>> plot(x[:,0],’b’,
label = ’Series 1’)
>>> hold(True)
>>> plot(x[:,1],’g.’,
label = ’Series 2’)
>>> plot(x[:,2],’r:’,label = ’Series 3’)
>>> legend(loc = 0, frameon = False, title = ’The Legend’)
>>> title(’Improved Legend’)
\end{verbatim}
\end{framed}

\newpage
\subsection{Plotting}
% - http://nbviewer.ipython.org/github/twiecki/financial-analysis-python-tutorial/blob/master/1.%20Pandas%20Basics.ipynb

\begin{framed}
\begin{verbatim}
close_px.plot(label='AAPL')
mavg.plot(label='mavg')
plt.legend()

\end{verbatim}
\end{framed}
%--------------------------------------------------------------------------------------------%
\end{document}
